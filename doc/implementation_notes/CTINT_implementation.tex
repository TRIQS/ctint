%% LyX 2.3.2 created this file.  For more info, see http://www.lyx.org/.
%% Do not edit unless you really know what you are doing.
\documentclass[english]{article}
\usepackage[utf8]{inputenc}
\usepackage[a4paper]{geometry}
\geometry{verbose,tmargin=2.5cm,bmargin=2.5cm,lmargin=2cm,rmargin=2cm}
\setlength{\parskip}{\smallskipamount}
\setlength{\parindent}{0pt}
\usepackage{color}
\definecolor{note_fontcolor}{rgb}{1, 0, 0}
\usepackage{amsmath}
\usepackage{amssymb}
\usepackage{graphicx}
\usepackage[unicode=true,pdfusetitle,
 bookmarks=true,bookmarksnumbered=false,bookmarksopen=false,
 breaklinks=true,pdfborder={0 0 1},backref=false,colorlinks=true]
 {hyperref}
\hypersetup{
 pdfborderstyle=}

\makeatletter

%%%%%%%%%%%%%%%%%%%%%%%%%%%%%% LyX specific LaTeX commands.
%% The greyedout annotation environment
\newenvironment{lyxgreyedout}
  {\textcolor{note_fontcolor}\bgroup\ignorespaces}
  {\ignorespacesafterend\egroup}

%%%%%%%%%%%%%%%%%%%%%%%%%%%%%% User specified LaTeX commands.

\usepackage{color}

\newcommand{\comment}[1]{\textcolor{blue}{\textbf{[Comment: #1]}}}
\newcommand{\n}{\tilde{n}}
\newcommand{\T}{{{\cal T}\,}}

\newcommand{\up}{\uparrow}
\newcommand{\dn}{\downarrow}
\newcommand{\upup}{\uparrow\uparrow}
\newcommand{\updn}{\uparrow\downarrow}
\newcommand{\dnup}{\downarrow\uparrow}
\newcommand{\dndn}{\downarrow\downarrow}
\newcommand{\xph}{{ \overline{ph} }}			% Exchange Particle-hole

\newcommand{\sign}{\operatorname{sign}}
\newcommand{\mc}[1]{\mathcal{#1}}

\newcommand{\Gt}{\tilde{\cal G}}
\newcommand{\Gc}{{\cal G}}

\DeclareMathOperator*{\sumint}{
   \mathchoice
   {\ooalign{$\displaystyle\sum$\cr\hidewidth$\displaystyle\int$\hidewidth\cr}}
   {\ooalign{\raisebox{.14\height}{\scalebox{.7}{$\textstyle\sum$}}\cr\hidewidth$\textstyle\int$\hidewidth\cr}}
   {\ooalign{\raisebox{.2\height}{\scalebox{.6}{$\scriptstyle\sum$}}\cr$\scriptstyle\int$\cr}}
   {\ooalign{\raisebox{.2\height}{\scalebox{.6}{$\scriptstyle\sum$}}\cr$\scriptstyle\int$\cr}}
}

\newcommand{\ceq}[1]{Eq.~(\ref{eq:#1})}
\newcommand{\cfg}[1]{Fig.~\ref{fig:#1}}

%opening
\title{Cluster CTINT with dynamic interactions - implementation notes}

\makeatother

\begin{document}
\title{CTINT: derivation and implementation notes}

\maketitle
\tableofcontents{}

\pagebreak{}

\section{The CTINT algorithm}

The following section describes the CTINT algorithm and its derivation.
We first present it for the case of static density-density interactions,
and present the extension to general dynamical interactions thereafter.
For notations and conventions used throughout these notes, the reader
is referred to the attached document (which we will refer to as ``vertex
conventions'').

\subsection{Action with static density-density interaction}

\label{subsec:static_dens}

Let us begin by considering the example of an impurity system with
static density-density interaction. The action can then be written
as 
\begin{equation}
\mc{S}=\mc{S}_{0}+S_{\mathrm{int}}
\end{equation}

with 
\begin{equation}
\mc{S}_{0}=-\bar{c}_{\bar{\alpha}}\Gc_{\bar{\alpha}\beta}^{-1}c_{\beta}
\end{equation}

and 
\begin{equation}
\mc{S}_{\mathrm{int}}=\sum_{ab}U_{ab}\int_{\tau}n_{a}(\tau)n_{b}(\tau).
\end{equation}


\subsection{Interaction expansion}

In CT-INT we proceed by expanding the partition function 
\begin{equation}
Z\equiv\int\mc{D}[\bar{c},c]e^{\bar{c}_{\bar{\alpha}}\Gc_{\bar{\alpha}\beta}^{-1}c_{\beta}-\mc{S}_{\mathrm{int}}}\label{eq:partition_function_def}
\end{equation}

in powers of $\mc{S}_{\mathrm{int}}$. One obtains 
\begin{equation}
Z=Z_{0}\sum_{k=0}^{\infty}\frac{(-1)^{k}}{k!}\sum_{\mathbf{a},\mathbf{b}}\left(\prod_{i=1}^{k}U_{a_{i}b_{i}}\right)\int_{\boldsymbol{\tau}}\langle n_{a_{1}}(\tau_{1})\dots n_{a_{k}}(\tau_{k})n_{b_{1}}(\tau_{1})\dots n_{b_{k}}(\tau_{k})\rangle_{0},\label{eq:Z_full}
\end{equation}
where here, and in the following, bold symbols denote vectors. Note
that in practice one typically orders the integral times, which allows
to cancel the $1/k!$ prefactor. Using the Wick theorem we can write
the non-interacting expectation value in the equation above compactly
using a determinant 
\begin{equation}
Z=Z_{0}\sum_{k=0}^{\infty}\frac{(-1)^{k}}{k!}\sum_{\mathbf{a},\mathbf{b}}\left(\prod_{i=1}^{k}U_{a_{i}b_{i}}\right)\int_{\boldsymbol{\tau}}\det\hat{{\cal G}}_{k},
\end{equation}
Here we have introduced the $2k\times2k$ matrix 
\begin{equation}
[\hat{{\cal G}}_{k}]_{ij}=\Gc_{x_{i}\bar{y}_{j}}.
\end{equation}
with vectors 
\begin{align}
{\boldsymbol{x}}[\mathbf{a},\mathbf{b},{\boldsymbol{\tau}}] & \equiv[(c_{1},\tau_{1}),\ldots,(c_{2k},\tau_{2k})]\equiv[(a_{1},\tau_{1}),\ldots,(a_{k},\tau_{k}),(b_{1},\tau_{1}),\ldots,(b_{k},\tau_{k})],\\
\bar{\boldsymbol{y}}[\mathbf{a},\mathbf{b},{\boldsymbol{\tau}}] & \equiv[(\bar{c}_{1},\tau_{1}),\ldots,(\bar{c}_{2k},\tau_{2k})]\equiv[(a_{1},\tau_{1}),\ldots,(a_{k},\tau_{k}),(b_{1},\tau_{1}),\ldots,(b_{k},\tau_{k})].
\end{align}
In summary we can write 
\begin{eqnarray}
Z & = & Z_{0}\Xi\label{eq:Z_as_Z0Xi}
\end{eqnarray}
with 
\begin{equation}
\Xi\equiv\sum_{k=0}^{\infty}\sum_{\mathbf{a},\mathbf{b}}\int_{\boldsymbol{\tau}}A_{k}[\mathbf{a},\mathbf{b}]\times\det\hat{{\cal G}}_{k}[\mathbf{a},\mathbf{b},{\boldsymbol{\tau}}]\label{eq:Xi_def}
\end{equation}

and 
\begin{equation}
A_{k}[\mathbf{a},\mathbf{b}]=\frac{(-1)^{k}}{k!}\left(\prod_{i=1}^{k}U_{a_{i}b_{i}}\right)
\end{equation}

Note that $A_{k}$ is dependent on the interaction $U$, while the
determinant is dependent on the bare propagator ${\cal G}$.

\subsection{The $\alpha$ function and the auxiliary spin}

In order to handle the trivial sign problem in the case of a static
density-density interaction, the use of a shifted density 
\begin{equation}
\tilde{n}_{as}(\tau)\equiv n_{a}(\tau)-\alpha_{as}.\label{eq:n_tilde_def}
\end{equation}
has proven useful\footnote{Indeed, in the particle hole symmetric case, a choice $\alpha_{a}=1/2$
allows to avoid the sign-problem afterall.}. Here, $\alpha_{as}$ is the shift-parameter, and $s\in\{\uparrow,\downarrow\}$
denotes an auxiliary spin. If this shift is accounted for in both
the kinetic and the interaction term of the action, the overall action
remains unchanged, i.e. 
\begin{equation}
\mc{S}=\tilde{{\cal S}}_{0}+\tilde{{\cal S}}_{\mathrm{int}},
\end{equation}
with 
\begin{equation}
\tilde{{\cal S}}_{0}=-\bar{c}_{\bar{\alpha}}\Gt_{\bar{\alpha}\beta}^{-1}c_{\beta}
\end{equation}
and 
\begin{equation}
\tilde{{\cal S}}_{\mathrm{int}}=\sum_{ab}\frac{U_{ab}}{N_{s}}\int_{\tau}\sum_{s}\tilde{n}_{as}(\tau)\tilde{n}_{bs}(\tau).
\end{equation}

We also defined the shifted non-interacting propagator 
\begin{equation}
\Gt_{\bar{a}b}^{-1}(i\omega)\equiv\Gc_{\bar{a}b}^{-1}(i\omega)-\delta_{\bar{a},b}\sum_{c}\frac{U_{bc}+U_{cb}}{N_{s}}\sum_{s}\alpha_{cs}.\label{eq:Gtilde_def}
\end{equation}

Note that the task of the auxiliary spin is to avoid an explicit symmetry
breaking in the non-interacting part, which requires that the symmetries
$\alpha_{(u,\uparrow),\downarrow}=\alpha_{(u,\downarrow),\uparrow}$
and $\alpha_{(u,\uparrow),\uparrow}=\alpha_{(u,\downarrow),\downarrow}$
are fullfilled.

While in principle the choice of $\alpha_{as}$ is not limited further,
experience suggests the common choice $\alpha_{as}=A_{a}+f_{as}\delta$,
where $A_{a}\sim{\cal O}(1/2)$, $\delta$ some small but positive
number, and a function $f_{as}:\{as\}\rightarrow\{-1,1\}$ fullfilling
the aforementioned symmetry. The path-integral of $Z$ has to be adjusted
accordingly 
\begin{equation}
Z=Z_{0}\sum_{k=0}^{\infty}\frac{(-1)^{k}}{k!N_{s}^{k}}\sum_{\mathbf{a},\mathbf{b},\mathbf{s}}\left(\prod_{i=1}^{k}U_{a_{i}b_{i}}\right)\int_{\boldsymbol{\tau}}\det\hat{{\cal G}}_{k},
\end{equation}
with matrix 
\begin{equation}
[\hat{{\cal G}}_{k}]_{ij}=\Gt_{x_{i}\bar{y}_{j}}-\delta_{i,j}\,\alpha_{c_{i}s_{i}}.\label{eq:det_alpha}
\end{equation}

To see this, one has to rewrite an expectation value of the kind $\langle\tilde{n}_{a_{1}s_{1}}(\tau_{1})\dots\tilde{n}_{b_{k}s_{k}}(\tau_{k})\rangle$
into a determinant. This procedure is again based on the Wick theorem,
and leads to the matrix in \ceq{detalpha}, as can be easily proven,
starting from the expression $\langle n_{a_{1}s_{1}}(\tau_{1})\dots n_{b_{k}s_{k}}(\tau_{k})\rangle$,
by successive replacements $n\rightarrow\tilde{n}$. We then get 
\begin{equation}
\Xi\equiv\sum_{k=0}^{\infty}\sum_{\mathbf{a},\mathbf{b},\mathbf{s}}\int_{\boldsymbol{\tau}}A_{k}[\mathbf{a},\mathbf{b}]\times\det\hat{{\cal G}}_{k}[\mathbf{a},\mathbf{b},\mathbf{s},{\boldsymbol{\tau}}]\label{eq:Xi_def}
\end{equation}
and 
\begin{equation}
A_{k}[\mathbf{a},\mathbf{b}]=\frac{(-1)^{k}}{k!N_{s}^{k}}\left(\prod_{i=1}^{k}U_{a_{i}b_{i}}\right)
\end{equation}


\subsection{The $\alpha$ function and the auxiliary spin}

In order to handle the trivial sign problem originating from the minus-signs
associated with the expansion of each static interaction vertex we
shift the quartic part of the action $\mc{S}_{\mathrm{{int}}}\rightarrow\mc{\tilde{S}}_{\mathrm{{int}}}$
with
\begin{eqnarray*}
\tilde{{\cal S}}_{\mathrm{int}} & = & \sum_{l}U_{l}\int_{\tau}\bar{c}_{\bar{a}_{l}}(\tau)c_{b_{l}}(\tau)\bar{c}_{\bar{c}_{l}}(\tau)c_{d_{l}}(\tau)-n_{\bar{a}_{l}b_{l}}(\tau)\alpha_{11}^{l}-n_{\bar{c}_{l}d_{l}}(\tau)\alpha_{00}^{l}+n_{\bar{a}_{l}d_{l}}(\tau)\alpha_{10}^{l}+n_{\bar{c}_{l}b_{l}}(\tau)\alpha_{01}^{l}\\
\\
 & = & \sum_{l}U_{l}\int_{\tau}\left(n_{\bar{a}_{l}b_{l}}(\tau)-\alpha_{00}^{l}\right)\left(n_{\bar{c}_{l}d_{l}}(\tau)-\alpha_{11}^{l}\right)+n_{\bar{a}_{l}d_{l}}(\tau)\alpha_{10}^{l}+n_{\bar{c}_{l}b_{l}}(\tau)\alpha_{01}^{l}+\mathrm{{const}}\\
 & = & \sum_{l}U_{l}\int_{\tau}-\left(n_{\bar{a}_{l}d_{l}}(\tau)-\alpha_{01}^{l}\right)\left(n_{\bar{c}_{l}b_{l}}(\tau)-\alpha_{10}^{l}\right)-n_{\bar{a}_{l}b_{l}}(\tau)\alpha_{11}^{l}-n_{\bar{c}_{l}d_{l}}(\tau)\alpha_{00}^{l}+\mathrm{{const}}
\end{eqnarray*}

This quadratic shift has to be accounted for in both the kinetic and
the interaction term of the action such that the overall action remains
unchanged, i.e. 
\begin{equation}
\mc{S}=\tilde{{\cal S}}_{0}+\tilde{{\cal S}}_{\mathrm{int}},
\end{equation}
with 
\begin{equation}
\tilde{{\cal S}}_{0}=-\bar{c}_{\bar{\alpha}}\Gt_{\bar{\alpha}\beta}^{-1}c_{\beta}.
\end{equation}

Here we introduced the non-interacting propagator by the quadratic
cross terms of the product above 
\begin{equation}
\Gt_{\bar{e}f}^{-1}(i\omega)\equiv\Gc_{\bar{e}f}^{-1}(i\omega)-\sum_{l}U_{l}\left(\delta_{\bar{e},\bar{a}_{l}}\delta_{f,b_{l}}\alpha_{11}^{l}+\delta_{\bar{e},\bar{c}_{l}}\delta_{f,d_{l}}\alpha_{00}^{l}-\delta_{\bar{e},\bar{a}_{l}}\delta_{f,d_{l}}\alpha_{10}^{l}-\delta_{\bar{e}\bar{c}_{l}}\delta_{fb_{l}}\alpha_{01}^{l}\right)
\end{equation}

The path-integral of $Z$ has to be adjusted accordingly 
\begin{equation}
Z=Z_{0}\sum_{k=0}^{\infty}\frac{(-1)^{k}}{k!}\sum_{\mathbf{l}}\left(\prod_{i=1}^{k}U_{l_{i}}\right)\int_{\boldsymbol{\tau}}\det\hat{{\cal G}}_{\mathbf{l}}[\boldsymbol{\tau}],
\end{equation}
with matrix 
\begin{equation}
\hat{{\cal G}}_{\mathbf{l}}[\boldsymbol{\tau}]=\begin{pmatrix}\tilde{G}_{\mathbf{b_{l}\bar{a}_{l}}}[\boldsymbol{\tau}]-\mathrm{diag}(\boldsymbol{\alpha_{00}^{\mathbf{l}}}) &  & \tilde{G}_{\mathbf{b_{l}\bar{c}_{l}}}[\boldsymbol{\tau}]-\mathrm{diag}(\boldsymbol{\alpha}_{\mathbf{10}}^{\mathbf{l}})\\
\tilde{G}_{\mathbf{d_{l}\bar{a}_{l}}}[\boldsymbol{\tau}]-\mathrm{diag}(\boldsymbol{\alpha}_{\mathbf{01}}^{\mathbf{l}}) &  & \tilde{G}_{\mathbf{d_{l}\bar{c}_{l}}}[\boldsymbol{\tau}]-\mathrm{diag}(\boldsymbol{\alpha_{11}^{l}})
\end{pmatrix}\label{eq:det_alpha}
\end{equation}

To see this, one has to rewrite an expectation value of the kind $\langle\tilde{n}_{a_{l_{1}}}(\tau_{1})\dots\tilde{n}_{b_{l_{k}}}(\tau_{k})\rangle$
into a determinant. This procedure is again based on the Wick theorem,
and leads to the matrix in \ceq{detalpha}, as can be easily proven,
starting from the expression $\langle n_{a_{l_{k}}}(\tau_{1})\dots n_{b_{l_{k}}}(\tau_{k})\rangle$,
by successive replacements $n\rightarrow\tilde{n}$. We then get 
\begin{equation}
\Xi\equiv\sum_{k=0}^{\infty}\sum_{\mathbf{l}}\int_{\boldsymbol{\tau}}A_{\mathbf{l}}\times\det\hat{{\cal G}}_{\mathbf{l}}[{\boldsymbol{\tau}}]\label{eq:Xi_def}
\end{equation}
and 
\begin{equation}
A_{\mathbf{l}}=\frac{(-1)^{k}}{k!}\left(\prod_{i=1}^{k}U_{l_{i}}\right)
\end{equation}

We solve the following set of equations self-consistently
\begin{align}
\Gt_{b_{l}\bar{a}_{l}}[\hat{\alpha}^{\mathbf{l}}] & =\alpha_{00}^{l}\\
\Gt_{b_{l}\bar{c}_{l}}[\hat{\alpha}^{\mathbf{l}}] & =\alpha_{10}^{l}\\
\Gt_{d_{l}\bar{a}_{l}}[\hat{\alpha}^{\mathbf{l}}] & =\alpha_{01}^{l}\\
\Gt_{d_{l}\bar{d}_{l}}[\hat{\alpha}^{\mathbf{l}}] & =\alpha_{11}^{l}
\end{align}
in order to determine alpha based off of the self-consistent Hartree
Fock solution. We finally introduce a small term-wise assymetry to
cope with the trivial part of the sign problem

\[
\hat{\alpha}_{l}\rightarrow\hat{\alpha}_{l}+\begin{pmatrix}\delta & \delta\\
\delta & -\delta
\end{pmatrix}
\]


\subsection{General dynamic interactions}

Let us now consider an interaction of the most general form 
\begin{align}
\mc{S}_{\mathrm{int}} & =\sumint_{\bar{\alpha}\beta\bar{\gamma}\delta}U_{\bar{\alpha}\beta\bar{\gamma}\delta}\,\bar{c}_{\alpha}c_{\beta}\bar{c}_{\gamma}c_{\delta}\\
 & =\sum_{\bar{a}b\bar{c}d}\int_{\boldsymbol{\tau}}U_{\bar{a}b\bar{c}d}(\tau_{a},\tau_{b},\tau_{c},\tau_{d})\times\bar{c}_{a}(\tau_{a})c_{b}(\tau_{b})\bar{c}_{c}(\tau_{c})c_{d}(\tau_{d}).
\end{align}
In this case the expansion of the partition function takes the form
\begin{equation}
Z=Z_{0}\sum_{k=0}^{\infty}\frac{(-1)^{k}}{k!}\sumint_{\bar{\boldsymbol{\alpha}}{\boldsymbol{\beta}}\bar{\boldsymbol{\gamma}}{\boldsymbol{\delta}}}\left(\prod_{i=1}^{k}U_{\bar{\alpha}_{i}\beta_{i}\bar{\gamma}_{i}\delta_{i}}\right)\langle\bar{c}_{\alpha_{1}}c_{\beta_{1}}\bar{c}_{\gamma_{1}}c_{\delta_{1}}\ldots\bar{c}_{\alpha_{k}}c_{\beta_{k}}\bar{c}_{\gamma_{k}}c_{\delta_{k}}\rangle_{0},\label{eq:Z_full}
\end{equation}
which again, using the Wick theorem, takes the compact form 
\begin{equation}
Z=Z_{0}\sum_{k=0}^{\infty}\frac{(-1)^{k}}{k!}\sumint_{\bar{\boldsymbol{\alpha}}{\boldsymbol{\beta}}\bar{\boldsymbol{\gamma}}{\boldsymbol{\delta}}}\left(\prod_{i=1}^{k}U_{\bar{\alpha}_{i}\beta_{i}\bar{\gamma}_{i}\delta_{i}}\right)\det\hat{{\cal G}}_{k},
\end{equation}
with a matrix 
\begin{equation}
[\hat{{\cal G}}_{k}]_{ij}=\Gc_{x_{i}\bar{y}_{j}},
\end{equation}
that now has indices 
\begin{align}
{\boldsymbol{x}}[{\boldsymbol{\beta}},{\boldsymbol{\delta}}]\equiv[{\beta_{1}}{\delta_{1}}\ldots{\beta_{k}}{\delta_{k}}],\qquad\bar{\boldsymbol{y}}[\bar{\boldsymbol{\alpha}},\bar{\boldsymbol{\gamma}}]\equiv[\bar{\alpha}_{1}\bar{\gamma}_{1}\ldots\bar{\alpha}_{k}\bar{\gamma}_{k}].
\end{align}

In summary we find that 
\begin{equation}
\Xi=\frac{Z}{Z_{0}}=\sum_{k=0}^{\infty}\sumint_{\bar{\boldsymbol{\alpha}}{\boldsymbol{\beta}}\bar{\boldsymbol{\gamma}}{\boldsymbol{\delta}}}A_{k}[{\boldsymbol{\alpha}},{\boldsymbol{\beta}},{\boldsymbol{\gamma}},{\boldsymbol{\delta}}]\times\det\hat{{\cal G}}_{k}[{\boldsymbol{\alpha}},{\boldsymbol{\beta}},{\boldsymbol{\gamma}},{\boldsymbol{\delta}},{\boldsymbol{s}}]\label{eq:qmc_sum}
\end{equation}
with 
\begin{equation}
A_{k}[{\boldsymbol{\alpha}},{\boldsymbol{\beta}},{\boldsymbol{\gamma}},{\boldsymbol{\delta}}]=\frac{(-1)^{k}}{k!}\left(\prod_{i=1}^{k}U_{\bar{\alpha}_{i}\beta_{i}\bar{\gamma}_{i}\delta_{i}}\right).
\end{equation}
\ceq{qmcsum} can be understood as a sum over weights $w_{{\cal C}}$
for all configurations ${\cal C}$ 
\begin{equation}
\sum_{{\cal C}}w_{{\cal C}}
\end{equation}


\subsection{Dynamic density-density and spin-exchange interactions}

Let us now consider retarded interactions of both the density-density
($\mc{D}$) and the spin-exchange ($\mc{J}^{\perp}$) type. In this
case the interacting part of the action reads 
\begin{eqnarray}
\mc{S}_{\mathrm{int}} & = & \frac{1}{N_{s}}\iint_{\tau,\tau'}\sum_{abs}\mc{D}_{ab}(\tau-\tau')\tilde{n}_{as}(\tau)\tilde{n}_{bs}(\tau')\\
 &  & +\frac{1}{2}\iint_{\tau,\tau'}\sum_{uv}{\cal J}_{uv}^{\perp}(\tau-\tau')\left[s_{u}^{+}(\tau)s_{v}^{-}(\tau')+s_{u}^{-}(\tau)s_{v}^{+}(\tau')\right]
\end{eqnarray}
with 
\begin{align}
s_{u}^{+}(\tau) & =\bar{c}_{u,\uparrow}(\tau)c_{u,\downarrow}(\tau),\\
s_{u}^{-}(\tau) & =\bar{c}_{u,\downarrow}(\tau)c_{u,\uparrow}(\tau).
\end{align}
As done already in Sec.~\ref{subsec:static_dens}, we have to adjust
the non-interacting propagator due to the use of the shifted densities
$\tilde{n}$ 
\begin{equation}
\Gt_{\bar{a}b}^{-1}(i\omega)\equiv\Gc_{\bar{a}b}^{-1}(i\omega)-\delta_{\bar{a},b}\sum_{c}\frac{1}{N_{s}}\left[\mc{D}_{bc}(i\Omega=0)+\mc{D}_{cb}(i\Omega=0)\right]\sum_{s}\alpha_{cs}.\label{eq:Gtilde_def}
\end{equation}


\subsection{Monte-Carlo algorithm}

The sum in \ceq{Xidef} is evaluated using the Metropolis-Hastings
algorithm. It can be understood as a sum over configurations $\mc{C}\equiv(k,\mathbf{a},\mathbf{b},{\boldsymbol{\tau}})$
(and possibly ${\bf s}$) with different probabilities $|w_{{\cal C}}|$,
i.e. 
\[
\Xi=\sum_{{\cal C}}w_{{\cal C}}=\sum_{{\cal C}}|w_{{\cal C}}|\sign(\mc{C}).
\]

The weights $|w_{{\cal C}}|$ are used to construct a Markov chain
of configurations satisfying detailed balance and ergodicity. Under
these conditions, the expectation value of an observable $\hat{O}$
can be computed as: 
\begin{equation}
\langle\hat{O}\rangle=\frac{\sum_{{\cal C}}|w_{{\cal C}}|\sign(\mc{C})\hat{O}_{{\cal C}}}{\sum_{{\cal C}}|w_{{\cal C}}|\sign(\mc{C})}=\frac{\langle\sign\hat{O}\rangle_{\mathrm{MC}}}{\langle\sign\rangle_{\mathrm{MC}}}\label{eq:MC_average_def}
\end{equation}

where we have defined 
\begin{equation}
\langle\hat{A}\rangle_{\mathrm{MC}}\equiv\sum_{{\cal C}}|w_{{\cal C}}|\hat{A}_{{\cal C}}.\label{eq:MC_avg_def}
\end{equation}

Here, $\hat{A}_{{\cal C}}$ is the value of observable $\hat{A}$
when evaluated in the MC configuration $\mc{C}$.

\section{Correlation functions and measurements}

In this section we discuss the measurement of different correlation
functions in CTInt. Note that in the following we will not explicitly
denote that the propagator has been shifted using a $\Gt$, but we
will rather use the symbol ${\cal G}$.

\subsection{Single-particle Green's function $G$}

From \ceq{partitionfunctiondef} we see that $\frac{\partial Z}{\partial\Gc_{\bar{\beta}\alpha}^{-1}}=Z\langle\bar{c}_{\bar{\beta}}c_{\alpha}\rangle$,
i.e. 
\begin{equation}
G_{\alpha\bar{\beta}}=\frac{1}{Z}\frac{\partial Z}{\partial\Gc_{\bar{\beta}\alpha}^{-1}}\label{eq:G_as_derivative_of_Z}
\end{equation}

Using Eqs.~(\ref{eq:G_as_derivative_of_Z}),(\ref{eq:Z_as_Z0Xi})
and (\ref{eq:Xi_def}) this can be further evaluated to 
\begin{align}
G_{\alpha\bar{\beta}}=\frac{1}{Z}\left[\frac{\partial Z_{0}}{\partial\Gc_{\bar{\beta}\alpha}^{-1}}\Xi+Z_{0}\frac{\partial\Xi}{\partial\Gc_{\bar{\beta}\alpha}^{-1}}\right]=\frac{1}{Z}\left[\Gc_{\alpha\bar{\beta}}Z_{0}\Xi-Z_{0}\Xi\Gc_{\alpha\bar{\delta}}\frac{1}{\Xi}\frac{\partial\Xi}{\partial\Gc_{\gamma\bar{\delta}}}\Gc_{\gamma\bar{\beta}}\right].\label{eq:G_interm}
\end{align}
Hence 
\begin{equation}
G_{\alpha\bar{\beta}}=\Gc_{\alpha\bar{\beta}}+\Gc_{\alpha\bar{\delta}}M_{\bar{\delta}\gamma}\Gc_{\gamma\bar{\beta}},\label{eq:G_vs_M}
\end{equation}
with 
\begin{equation}
M_{\bar{\alpha}\beta}\equiv-\frac{1}{\Xi}\frac{\partial\Xi}{\partial\Gc_{\beta\bar{\alpha}}}.\label{eq:M_def}
\end{equation}
Instead of measuring $G$ directly, one typically measures the quantity
$M_{\bar{\alpha}\beta}$ instead, as is done in the CTINT code of
TRIQS. Let us see how it can be written by means of a Monte-Carlo
average. \ceq{Xidef} gives 
\[
M_{\bar{\alpha}\beta}=-\frac{1}{\Xi}\sum_{k=0}^{\infty}\sum_{\mathbf{a},\mathbf{b},\mathbf{s}}\int_{\boldsymbol{\tau}}A_{k}\frac{\partial\det\hat{{\cal G}}_{k}}{\partial\Gc_{\beta\bar{\alpha}}}.
\]

Using the chain rule 
\[
\frac{\partial\det\hat{{\cal G}}_{k}}{\partial\Gc_{\beta\bar{\alpha}}}=\sum_{ij}\frac{\partial\det\hat{{\cal G}}_{k}}{\partial[\mc{\hat{G}}_{k}]_{ij}}\frac{\partial[\hat{{\cal G}}_{k}]_{ij}}{\partial\Gc_{\beta\bar{\alpha}}}=\det\hat{\Gc}_{k}\sum_{ij}[\hat{{\cal G}}_{k}^{-1}]_{ji}\delta_{x_{i}\beta}\delta_{\bar{y}_{i}\bar{\alpha}}
\]
we obtain 
\begin{align}
M_{\bar{\alpha}\beta} & =-\frac{1}{\Xi}\sum_{k=0}^{\infty}\sum_{\mathbf{a},\mathbf{b},\mathbf{s}}\int_{\boldsymbol{\tau}}A_{k}\frac{\partial\det\hat{{\cal G}}_{k}}{\partial\Gc_{\beta\bar{\alpha}}}\nonumber \\
 & =-\frac{1}{\Xi}\sum_{{\cal C}}w_{{\cal C}}\left\{ \sum_{ij}\delta_{\bar{y}_{i}\bar{\alpha}}\delta_{x_{i}\beta}[\hat{{\cal G}}_{k}^{-1}]_{ji}\right\} \nonumber \\
 & =\frac{\Big\langle-\sum_{ij}\delta_{\bar{y}_{i}\bar{\alpha}}\delta_{x_{i}\beta}[\hat{{\cal G}}_{k}^{-1}]_{ji}\,\sign(\mc{C})\Big\rangle_{\mathrm{MC}}}{\Big\langle\sign(\mc{C})\Big\rangle_{\mathrm{MC}}}.\label{eq:M_MC_final}
\end{align}


\paragraph{Imaginary time measurement}

\mbox{%
%
} \\[1.5ex] Just like the Green function, $M_{\bar{\alpha}\beta}$
depends only on the time-difference $\tau=\tau_{\bar{\alpha}}-\tau_{\beta}$,
which, in practice, is chosen from a discrete set of points $\{\tau_{l}\}$.
They are determined by binning the interval $[0,\beta)$ equidistantly,
such that grid points are seperated by $\Delta\tau$. We thus define
\begin{equation}
M_{\bar{a}b}^{\Delta\tau}(\tau_{l})=-\frac{1}{\beta\langle\sign({\cal C})\rangle_{\mathrm{MC}}}\left\langle \sum_{ij}\delta_{\bar{c}_{j}\bar{a}}\delta_{c_{i}b}\left[\delta_{\Delta\tau}(\tau_{l}-\tau_{j}+\tau_{i})-\delta_{\Delta\tau}(\tau_{l}-\beta-\tau_{j}+\tau_{i})\right][\hat{{\cal G}}_{k}^{-1}]_{ji}\sign({\cal C})\right\rangle _{\mathrm{MC}}
\end{equation}
where $\delta_{\Delta\tau}$ is the broadened delta function 
\begin{equation}
\delta_{\Delta\tau}(\tau)\equiv\begin{cases}
\frac{1}{\Delta\tau} & 0<\tau<\Delta\tau\\
0 & \tau\leq0\quad\mathrm{or}\quad\tau\geq\Delta\tau,
\end{cases}
\end{equation}
and a factor $1/\beta$ should be introduced when making use of time-translational
invariance $(\tau_{\bar{\alpha}},\tau_{\beta})\rightarrow\tau=\tau_{\bar{\alpha}}-\tau_{\beta}$.
The second $\delta$ function with the minus sign takes care of shifting
negative time-differences $\tau_{\bar{\alpha}}-\tau_{\beta}$ to the
interval $[0,\beta)$ using the antiperiodicity of $M$. As $M_{\bar{u}v}(\tau_{l})=\lim_{\Delta\tau\rightarrow0}{M}_{\bar{u}v}^{\Delta\tau}(\tau_{l})$
we can approximate $M_{\bar{u}v}(\tau_{l})\approx M_{\bar{u}v}^{\Delta\tau}(\tau_{l})$
for a sufficiently small $\Delta\tau$.

\paragraph{Frequency measurement}

\mbox{%
%
} \\[1.5ex] Alternatively, we can measure $M$ directly in frequency
space, i.e. 
\begin{align}
M_{\bar{a}b}(i\omega_{n})=-\frac{1}{\beta\langle\sign({\cal C})\rangle_{\mathrm{MC}}}\left\langle \sum_{ij}\delta_{\bar{c}_{j}\bar{a}}\delta_{c_{i}b}e^{i\omega_{n}(\tau_{j}-\tau_{i})}[\hat{{\cal G}}_{k}^{-1}]_{ji}\sign({\cal C})\right\rangle _{\mathrm{MC}}
\end{align}
Here, we can avoid the binning, while a factor $1/\beta$ has to be
introduced as before. To evaluate this measure, we can make use of
the NFFT library. To make this more obvious we rewrite 
\begin{align}
e^{i\omega_{n}(\tau_{j}-\tau_{i})}=e^{i\pi(2n+1)\left(\frac{\tau_{j}-\tau_{i}}{\beta}-\frac{1}{2}+\frac{1}{2}\right)}=e^{2\pi inx}e^{i\pi\left(n+x+\frac{1}{2}\right)}
\end{align}
with $x\equiv\left(\frac{\tau_{j}-\tau_{i}}{\beta}-\frac{1}{2}\right)\in[-0.5,0.5)$
and $n\in\mathbb{Z}$, as is required by the interface of the library.

%\subsection{Improved estimator $F$ }

%When we are measuring the improved estimator $F$, given as a function
%of $M$ by
%\[
%F_{\bar{\alpha}\bar{\beta}}=M_{\bar{\alpha}\gamma}{\cal G}_{\gamma\bar{\beta}}
%\]
%there is an additional integral over time which we do not do explicitly
%calculate, but sample also within the Monte carlo, i.e.
%\begin{eqnarray}
%F^{\Delta \tau}_{\bar{a}\bar{b}}(\tau_l) & = & -\frac{1}{\langle\sign({\cal C})\rangle_{\mathrm{MC}}}\left\langle \sum_{ij}\delta_{\bar{c}_j \bar{a}}\delta_{\Delta\tau}(\tau_l-\tau_j)[\hat{\cal G}_k^{-1}]_{ji}{\cal G}_{c_i\bar{b}}(\tau_i)\sign({\cal C})\right\rangle _{\mathrm{MC}}.
%\end{eqnarray}
%Note that the factor $1/\beta$ is not required here, since we did not make use of time-translational invariance here.
%The same measurement can again be performed in the frequency domain
%\begin{eqnarray}
%F_{\bar{a}\bar{b}}(i\omega_n) & = & -\frac{1}{\langle\sign({\cal C})\rangle_{\mathrm{MC}}}\left\langle \sum_{ij}\delta_{\bar{c}_j \bar{a}}\, e^{i\omega_n \tau_j}[\hat{\cal G}_k^{-1}]_{ji}{\cal G}_{c_i\bar{b}}(\tau_i)\sign({\cal C})\right\rangle _{\mathrm{MC}}.
%\end{eqnarray}

\subsection{Two-particle Green's function $\chi^{4}$\label{sec:Measuring-the-4-point}}

The same idea can be pursued when considering two-particle Green functions.
From Eq.~(\ref{eq:partition_function_def}) we can directly see that
$\frac{\partial Z}{\partial\Gc_{\bar{\alpha}\beta}^{-1}\partial\Gc_{\bar{\gamma}\delta}^{-1}}=Z\langle\bar{c}_{\alpha}c_{\beta}\bar{c}_{\gamma}c_{\delta}\rangle$
i.e 
\begin{equation}
\chi_{\bar{\alpha}\beta\bar{\gamma}\delta}^{4}=\langle\bar{c}_{\alpha}c_{\beta}\bar{c}_{\gamma}c_{\delta}\rangle=\frac{1}{Z}\frac{\partial Z}{\partial\Gc_{\bar{\alpha}\beta}^{-1}\partial\Gc_{\bar{\gamma}\delta}^{-1}}\label{eq:chi4_as_derivative_of_Z}
\end{equation}

Starting from Eqs.~(\ref{eq:chi4_as_derivative_of_Z}), (\ref{eq:G_as_derivative_of_Z})
and (\ref{eq:df_dG_vs_df_dinvG}), we first note:

\begin{equation}
\chi_{\bar{\alpha}\beta\bar{\eta}\kappa}^{4}=\frac{1}{Z}\frac{\partial(ZG_{\beta\bar{\alpha}})}{\partial\Gc_{\bar{\eta}\kappa}^{-1}}=-\frac{1}{Z}\Gc_{\mu\bar{\eta}}\frac{\partial(ZG_{\beta\bar{\alpha}})}{\partial\Gc_{\mu\bar{\nu}}}\Gc_{\kappa\bar{\nu}}\label{eq:chi4_interm}
\end{equation}

Let us now evaluate $\frac{\partial(ZG_{\beta\bar{\alpha}})}{\partial\Gc_{\mu\bar{\nu}}}$
using (\ref{eq:G_interm}). We have (using (\ref{eq:dZ0_dG})) 
\begin{align}
\frac{\partial(ZG_{\beta\bar{\alpha}})}{\partial\Gc_{\mu\bar{\nu}}} & =\frac{\partial}{\partial\Gc_{\mu\bar{\nu}}}\left[\Gc_{\beta\bar{\alpha}}Z_{0}\Xi-Z_{0}\Gc_{\gamma\bar{\alpha}}\frac{\partial\Xi}{\partial\Gc_{\gamma\bar{\delta}}}\Gc_{\beta\bar{\delta}}\right]\nonumber \\
 & =\delta_{\beta\mu}\delta_{\bar{\nu}\bar{\alpha}}Z_{0}\Xi+\Gc_{\beta\bar{\alpha}}\frac{\partial Z_{0}}{\partial\Gc_{\mu\bar{\nu}}}\Xi+\Gc_{\beta\bar{\alpha}}Z_{0}\frac{\partial\Xi}{\partial\Gc_{\mu\bar{\nu}}}\nonumber \\
 & \;\;-\frac{\partial Z_{0}}{\partial\Gc_{\mu\bar{\nu}}}\Gc_{\gamma\bar{\alpha}}\frac{\partial\Xi}{\partial\Gc_{\gamma\bar{\delta}}}\Gc_{\beta\bar{\delta}}-Z_{0}\delta_{\mu\gamma}\delta_{\bar{\alpha}\bar{\nu}}\frac{\partial\Xi}{\partial\Gc_{\gamma\bar{\delta}}}\Gc_{\beta\bar{\delta}}-Z_{0}\Gc_{\gamma\bar{\alpha}}\frac{\partial\Xi}{\partial\Gc_{\mu\bar{\nu}}\partial\Gc_{\gamma\bar{\delta}}}\Gc_{\beta\bar{\delta}}-Z_{0}\Gc_{\gamma\bar{\alpha}}\frac{\partial\Xi}{\partial\Gc_{\gamma\bar{\delta}}}\delta_{\beta\mu}\delta_{\bar{\delta}\bar{\nu}}\nonumber \\
 & =\delta_{\beta\mu}\delta_{\bar{\nu}\bar{\alpha}}Z_{0}\Xi-\Gc_{\beta\bar{\alpha}}\Gc_{\bar{\nu}\mu}^{-1}Z_{0}\Xi-\Gc_{\beta\bar{\alpha}}Z_{0}\Xi M_{\bar{\nu}\mu}\nonumber \\
 & \;\;-\Gc_{\bar{\nu}\mu}^{-1}Z_{0}\Gc_{\gamma\bar{\alpha}}\Xi M_{\bar{\delta}\gamma}\Gc_{\beta\bar{\delta}}+\Xi Z_{0}\delta_{\mu\gamma}\delta_{\bar{\alpha}\bar{\nu}}M_{\bar{\delta}\gamma}\Gc_{\beta\bar{\delta}}-Z_{0}\Gc_{\gamma\bar{\alpha}}\Xi M_{\bar{\nu}\mu\bar{\delta}\gamma}^{4}\Gc_{\beta\bar{\delta}}+Z_{0}\Gc_{\gamma\bar{\alpha}}\Xi M_{\bar{\delta}\gamma}\delta_{\beta\mu}\delta_{\bar{\delta}\bar{\nu}}\nonumber \\
 & =Z\Big\{\delta_{\beta\mu}\delta_{\bar{\nu}\bar{\alpha}}-\Gc_{\beta\bar{\alpha}}\Gc_{\bar{\nu}\mu}^{-1}-\Gc_{\beta\bar{\alpha}}M_{\bar{\nu}\mu}-\Gc_{\bar{\nu}\mu}^{-1}\Gc_{\gamma\bar{\alpha}}M_{\bar{\delta}\gamma}\Gc_{\beta\bar{\delta}}\nonumber \\
 & \;\;\;+\delta_{\bar{\alpha}\bar{\nu}}M_{\bar{\delta}\mu}\Gc_{\beta\bar{\delta}}-\Gc_{\gamma\bar{\alpha}}M_{\mu\bar{\nu}\gamma\bar{\delta}}^{4}\Gc_{\beta\bar{\delta}}+\Gc_{\gamma\bar{\alpha}}M_{\bar{\nu}\gamma}\delta_{\beta\mu}\Big\}\label{eq:chi4_interm_2}
\end{align}
where we have defined 
\begin{equation}
M_{\mu\bar{\nu}\gamma\bar{\delta}}^{4}\equiv\frac{1}{\Xi}\frac{\partial\Xi}{\partial\Gc_{\mu\bar{\nu}}\partial\Gc_{\gamma\bar{\delta}}}.\label{eq:M4_def}
\end{equation}

Plugging (\ref{eq:chi4_interm_2}) into (\ref{eq:chi4_interm}), we
obtain 
\begin{align*}
\chi_{\bar{\alpha}\beta\bar{\eta}\kappa}^{4} & =-\Gc_{\beta\bar{\eta}}\Gc_{\kappa\bar{\alpha}}+\Gc_{\kappa\bar{\eta}}\Gc_{\beta\bar{\alpha}}+\Gc_{\beta\bar{\alpha}}\Gc_{\kappa\bar{\nu}}M_{\bar{\nu}\mu}\Gc_{\mu\bar{\eta}}+\Gc_{\beta\bar{\delta}}M_{\bar{\delta}\gamma}\Gc_{\gamma\bar{\alpha}}\Gc_{\kappa\bar{\eta}}\\
 & \;\;-\Gc_{\beta\bar{\delta}}M_{\bar{\delta}\mu}\Gc_{\mu\bar{\eta}}\Gc_{\kappa\bar{\alpha}}+\Gc_{\mu\bar{\alpha}}\Gc_{\beta\bar{\nu}}M_{\mu\bar{\nu}\gamma\bar{\delta}}^{4}\Gc_{\gamma\bar{\eta}}\Gc_{\kappa\bar{\delta}}-\Gc_{\beta\bar{\eta}}\Gc_{\kappa\bar{\nu}}M_{\bar{\nu}\gamma}\Gc_{\gamma\bar{\alpha}}
\end{align*}
i.e, after relabeling and reordering, 
\begin{align}
\chi_{\bar{\alpha}\beta\bar{\gamma}\delta}^{4} & =\Gc_{\mu\bar{\alpha}}\Gc_{\beta\bar{\nu}}M_{\mu\bar{\nu}\kappa\bar{\eta}}^{4}\Gc_{\kappa\bar{\gamma}}\Gc_{\delta\bar{\eta}}\nonumber \\
 & \;\;+\Gc_{\beta\bar{\alpha}}\Gc_{\delta\bar{\gamma}}+[\Gc M\Gc]_{\beta\bar{\alpha}}\Gc_{\delta\bar{\gamma}}+\Gc_{\beta\bar{\alpha}}[\Gc M\Gc]_{\delta\bar{\gamma}}\nonumber \\
 & \;\;-\Gc_{\beta\bar{\gamma}}\Gc_{\delta\bar{\alpha}}-[\Gc M\Gc]_{\beta\bar{\gamma}}\Gc_{\delta\bar{\alpha}}-\Gc_{\beta\bar{\gamma}}[\Gc M\Gc]_{\delta\bar{\alpha}}.\label{eq:chi4_final}
\end{align}

Thus, in order to compute $\chi^{4}$, one needs to compute $M$ and
$M^{4}$.

\paragraph{Connected four-point function}

\mbox{%
%
} \\[1.5ex] By inspection of Eq.(\ref{eq:chi4_final}) we see that
the connected part of the two-particle Green function now reads 
\begin{align}
\chi_{\bar{\alpha}\beta\bar{\gamma}\delta}^{4,\mathrm{conn}}=\Gc_{\mu\bar{\alpha}}\Gc_{\beta\bar{\nu}}M_{\mu\bar{\nu}\kappa\bar{\eta}}^{4}\Gc_{\kappa\bar{\gamma}}\Gc_{\delta\bar{\eta}}-\left[\Gc M\Gc\right]_{\alpha\bar{\beta}}\left[\Gc M\Gc\right]_{\gamma\bar{\delta}}+\left[\Gc M\Gc\right]_{\alpha\bar{\delta}}\left[\Gc M\Gc\right]_{\gamma\bar{\beta}}.\label{eq:chi4tilde_conn_final}
\end{align}
Instead, we can write 
\[
\chi_{\bar{\alpha}\beta\bar{\gamma}\delta}^{4,\mathrm{conn}}=\Gc_{\mu\bar{\alpha}}\Gc_{\beta\bar{\nu}}M_{\mu\bar{\nu}\kappa\bar{\eta}}^{4,\mathrm{conn}}\Gc_{\kappa\bar{\gamma}}\Gc_{\delta\bar{\eta}}
\]
with 
\[
M_{\mu\bar{\nu}\kappa\bar{\eta}}^{4,\mathrm{conn}}\equiv M_{\mu\bar{\nu}\kappa\bar{\eta}}^{4}-M_{\bar{\nu}\mu}M_{\bar{\eta}\kappa}+M_{\bar{\nu}\kappa}M_{\bar{\eta}\mu}.
\]


\paragraph{Fully reducible vertex\label{par:Fully-reducible-vertex}}

\mbox{%
%
} \\[1.5ex] Using the definition of the fully reducible vertex function
$F$ (see vertex conventions), we get 
\[
F_{\alpha\bar{\beta}\gamma\bar{\delta}}=\bigl\{ G_{\bar{\beta}\beta}^{-1}{\cal G}_{\beta\bar{\nu}}\bigr\}\left\{ G_{\bar{\delta}\delta}^{-1}{\cal G}_{\delta\bar{\eta}}\right\} M_{\mu\bar{\nu}\kappa\bar{\eta}}^{4,\mathrm{conn}}\left\{ {\cal G}_{\mu\bar{\alpha}}G_{\bar{\alpha}\alpha}^{-1}\right\} \left\{ {\cal G}_{\kappa\bar{\gamma}}G_{\bar{\gamma}\gamma}^{-1}\right\} .
\]
With Eq.~(\ref{eq:G_vs_M}) we can calculate $F$ directly from $M^{4}$
and $M$ 
\[
F_{\alpha\bar{\beta}\gamma\bar{\delta}}=\left\{ 1+M{\cal G}\right\} _{\bar{\beta}\bar{\nu}}^{-1}\left\{ 1+M{\cal G}\right\} _{\bar{\delta}\bar{\eta}}^{-1}M_{\mu\bar{\nu}\kappa\bar{\eta}}^{4,\mathrm{conn}}\left\{ 1+M{\cal G}\right\} _{\mu\alpha}^{-1}\left\{ 1+M{\cal G}\right\} _{\kappa\gamma}^{-1}.
\]
From this expression, one can see that $F$ and $M^{4,\mathrm{conn}}$
have the same asymptotics.

\paragraph{Imaginary time measurement of $M^{4}$}

\mbox{%
%
} \\[1.5ex] In order to write $M^{4}$ by means of a Monte-Carlo average
we have to evaluate 
\begin{equation}
M_{\alpha\bar{\beta}\gamma\bar{\delta}}^{4}=\frac{1}{\Xi}\frac{\partial\Xi}{\partial\Gc_{\alpha\bar{\beta}}\partial\Gc_{\gamma\bar{\delta}}}=-\frac{1}{\Xi}\sum_{k=0}^{\infty}\sum_{\mathbf{a},\mathbf{b},\mathbf{s}}\int_{\boldsymbol{\tau}}A_{k}\,\frac{\partial\det\hat{{\cal G}}_{k}}{\partial\Gc_{\alpha\bar{\beta}}\partial\Gc_{\gamma\bar{\delta}}}.
\end{equation}
This derivative evaluates to (neglecting the $k$ index for now) 
\begin{align}
\frac{\partial\det\hat{{\cal G}}}{\partial\Gc_{\alpha\bar{\beta}}\partial\Gc_{\gamma\bar{\delta}}} & =\frac{\partial}{\partial\Gc_{\alpha\bar{\beta}}}\det\hat{{\cal G}}\,\sum_{kl}[\hat{{\cal G}}^{-1}]_{lk}\delta_{x_{k}\gamma}\delta_{\bar{y}_{l}\bar{\delta}}\\
 & =\det\hat{{\cal G}}\,\sum_{ijkl}\left\{ [\hat{{\cal G}}^{-1}]_{ji}[\hat{{\cal G}}^{-1}]_{lk}-[\hat{{\cal G}}^{-1}]_{li}[\hat{{\cal G}}^{-1}]_{jk}\right\} \delta_{x_{i}\alpha}\delta_{\bar{y}_{i}\bar{\beta}}\delta_{x_{k}\gamma}\delta_{\bar{y}_{l}\bar{\delta}},
\end{align}
where we had to use that 
\begin{align}
\phantom{=}\frac{\partial}{\partial\Gc_{\alpha\bar{\beta}}}\sum_{kl}[\hat{{\cal G}}^{-1}]_{lk}\delta_{x_{k}\gamma}\delta_{\bar{y}_{l}\bar{\delta}}=-\sum_{ijkl}[\hat{{\cal G}}^{-1}]_{li}\underbrace{\frac{\partial\hat{{\cal G}}_{ij}}{\partial\Gc_{\alpha\bar{\beta}}}}_{\delta_{x_{i}\alpha}\delta_{\bar{y}_{i}\bar{\beta}}}[\hat{{\cal G}}^{-1}]_{jk}\delta_{x_{k}\gamma}\delta_{\bar{y}_{l}\bar{\delta}}.
\end{align}
The Monte-Carlo average for $M^{4}$ thus takes the form 
\begin{align}
M_{\alpha\bar{\beta}\gamma\bar{\delta}}^{4}=\frac{1}{\langle\sign(\mc{C})\rangle_{\mathrm{MC}}}\Bigg\langle\sum_{ijkl}\left\{ [\hat{{\cal G}}^{-1}]_{ji}[\hat{{\cal G}}^{-1}]_{lk}-[\hat{{\cal G}}^{-1}]_{li}[\hat{{\cal G}}^{-1}]_{jk}\right\} \delta_{x_{i}\alpha}\delta_{\bar{y}_{j}\bar{\beta}}\delta_{x_{k}\gamma}\delta_{\bar{y}_{l}\bar{\delta}}\sign(\mc{C})\Bigg\rangle_{\mathrm{MC}}.\label{eq:Measure_M4}
\end{align}
Using the intermediate scattering matrix 
\begin{equation}
\bar{M}_{\bar{\beta}\alpha}\equiv\sum_{ij}[\hat{{\cal G}}^{-1}]_{ji}\delta_{x_{i}\alpha}\delta_{\bar{y}_{j}\bar{\beta}}
\end{equation}
it takes the compact form 
\begin{align}
M_{\alpha\bar{\beta}\gamma\bar{\delta}}^{4}=\frac{1}{\langle\sign(\mc{C})\rangle_{\mathrm{MC}}}\Big\langle\left(\bar{M}_{\bar{\beta}\alpha}\bar{M}_{\bar{\delta}\gamma}-\bar{M}_{\bar{\delta}\alpha}\bar{M}_{\bar{\beta}\gamma}\right)\sign(\mc{C})\Big\rangle_{\mathrm{MC}}.
\end{align}
This factorization is directly used in our implementation.

\paragraph{Frequency measurement of $M^{4}$}

\mbox{%
%
} \\[1.5ex] Just like $M$, we can measure $M^{4}$ also directly
in Matsubara frequencies\footnote{Careful: The Fourier conventions for $M$-objects are indeed reversed,
meaning that barred indices are transformed to frequencies with $\int_{\tau}e^{iw\tau}$.} 
\begin{equation}
\begin{split}M_{a\bar{b}c\bar{d}}^{4}(i\omega_{a},i\omega_{\bar{b}},i\omega_{c}) & =\frac{1}{\beta\langle\sign(\mc{C})\rangle_{\mathrm{MC}}}\Bigg\langle\left(\bar{M}_{\bar{\beta}\alpha}\bar{M}_{\bar{\delta}\gamma}-\bar{M}_{\bar{\delta}\alpha}\bar{M}_{\bar{\beta}\gamma}\right)\\
 & \hspace{2cm}\times e^{-i\omega_{a}(\tau_{a}-\tau_{d})}\,e^{i\omega_{\bar{b}}(\tau_{b}-\tau_{d})}\,e^{-i\omega_{c}(\tau_{c}-\tau_{d})}\sign(\mc{C})\Bigg\rangle_{\mathrm{MC}}\\
 & =\frac{1}{\beta\langle\sign(\mc{C})\rangle_{\mathrm{MC}}}\Bigg\langle\Big[\bar{M}_{\bar{b}a}(\omega_{b},\omega_{a})\bar{M}_{\bar{d}c}(\omega_{a}+\omega_{c}-\omega_{b},\omega_{c})\\
 & \hspace{2cm}-\bar{M}_{\bar{d}a}(\underbrace{\omega_{a}+\omega_{c}-\omega_{b}}_{\omega_{d}},\omega_{a})\bar{M}_{\bar{b}c}(\omega_{b},\omega_{c})\Big]\sign(\mc{C})\Bigg\rangle_{\mathrm{MC}}.
\end{split}
\end{equation}
Here we have defined the intermediate scattering matrix in the Matsubara
domain 
\begin{equation}
\bar{M}_{\bar{b}a}(\omega_{b},\omega_{a})\equiv\sum_{ij}[\hat{{\cal G}}^{-1}]_{ji}\delta_{c_{i}a}\delta_{\bar{c}_{j}\bar{b}}\times\,e^{i\omega_{b}\tau_{j}}e^{-i\omega_{a}\tau_{i}}.
\end{equation}
In the implementation we precompute this intermediate scattering matrix,
which allows us to reduce the effort from $\mc{O}(k^{4}\log(k))$
down to $\mc{O}(k^{2}\log(k))$.

\subsection{Equal-time correlation functions $\chi^{3}$\label{sec:Measuring-M3}}

The equal-time correlation functions $\chi^{3,r}$ can be directly
calculated from the intermediate quantity $M^{3,r}$ defined in the
following. In the first step we define three extensions of the intermediate
scattering matrix 
\begin{equation}
\overline{\Gc M}_{b\alpha}\equiv\overline{\Gc M}_{ba}(\tau_{a})\equiv\sum_{ij}\Gc_{b\bar{c}_{j}}(-\tau_{j})[\hat{\Gc}^{-1}]_{ji}\delta_{x_{i}\alpha}=-\sum_{ij}\Gc_{b\bar{c}_{j}}(\beta-\tau_{j})[\hat{\Gc}^{-1}]_{ji}\delta_{x_{i}\alpha},
\end{equation}
\begin{equation}
\overline{M\Gc}_{\bar{\beta}\bar{a}}\equiv\overline{M\Gc}_{\bar{b}\bar{a}}(\tau_{b})\equiv\sum_{ij}\delta_{y_{j}\beta}[\hat{\Gc}^{-1}]_{ji}\Gc_{c_{i}\bar{a}}(\tau_{i}),
\end{equation}
\begin{equation}
\overline{\Gc M\Gc}_{b\bar{a}}\equiv\sum_{ij}\Gc_{b\bar{c}_{j}}(-\tau_{j})[\hat{\Gc}^{-1}]_{ji}\Gc_{c_{i}\bar{a}}(\tau_{i})=-\sum_{ij}\Gc_{b\bar{c}_{j}}(\beta-\tau_{j})[\hat{\Gc}^{-1}]_{ji}\Gc_{c_{i}\bar{a}}(\tau_{i}).
\end{equation}
Using these, $M^{3,r}$ is measured as 
\begin{align}
M_{abcd}^{3,pp}(\tau_{a},\tau_{c})=\frac{1}{\langle\sign(\mc{C})\rangle_{\mathrm{MC}}}\Bigg\langle\Big[\overline{\Gc M}_{ba}(\tau_{a})\overline{\Gc M}_{dc}(\tau_{c})-\overline{\Gc M}_{da}(\tau_{a})\overline{\Gc M}_{bc}(\tau_{c})\Big]\sign(\mc{C})\Bigg\rangle_{\mathrm{MC}},
\end{align}
\begin{align}
M_{a\bar{b}\bar{c}d}^{3,ph}(\tau_{a},\tau_{\bar{b}})=\frac{1}{\langle\sign(\mc{C})\rangle_{\mathrm{MC}}}\Bigg\langle\Big[\bar{M}_{\bar{\beta}\alpha}\overline{\Gc M\Gc}_{d\bar{c}}-\overline{\Gc M}_{d\alpha}\overline{M\Gc}_{\bar{\beta}\bar{c}}\Big]\sign(\mc{C})\Bigg\rangle_{\mathrm{MC}}.
\end{align}


\paragraph{Frequency measurement of $M^{3,r}$}

\mbox{%
%
} \\[1.5ex] Again, we can measure $M^{3,r}$ also directly in Matsubara
frequencies. For this we need the scattering matrices 
\begin{equation}
\overline{\Gc M}_{ba}(\omega_{a})\equiv\sum_{ij}\Gc_{b\bar{c}_{j}}(-\tau_{j})[\hat{\Gc}^{-1}]_{ji}e^{-i\omega_{a}\tau_{i}}\delta_{c_{i}a}=\sum_{ij}\Gc_{b\bar{c}_{j}}(\beta-\tau_{j})[\hat{\Gc}^{-1}]_{ji}e^{i\omega_{a}(\beta-\tau_{i})}\delta_{c_{i}a},
\end{equation}
\begin{equation}
\overline{M\Gc}_{\bar{b}\bar{a}}(\omega_{\bar{b}})\equiv\sum_{ij}e^{i\omega_{\bar{b}}\tau_{j}}\delta_{\bar{c}_{j}\bar{b}}[\hat{\Gc}^{-1}]_{ji}\Gc_{c_{i}\bar{a}}(\tau_{i}),
\end{equation}
\begin{equation}
\overline{\Gc M\Gc}_{b\bar{a}}\equiv\sum_{ij}\Gc_{b\bar{c}_{j}}(-\tau_{j})[\hat{\Gc}^{-1}]_{ji}\Gc_{c_{i}\bar{a}}(\tau_{i})=-\sum_{ij}\Gc_{b\bar{c}_{j}}(\beta-\tau_{j})[\hat{\Gc}^{-1}]_{ji}\Gc_{c_{i}\bar{a}}(\tau_{i}),
\end{equation}

Then we can measure 
\begin{align}
M_{abcd}^{3,pp}(\omega_{a},\omega_{c})=\frac{1}{\langle\sign(\mc{C})\rangle_{\mathrm{MC}}}\Bigg\langle\Big[\overline{\Gc M}_{ba}(\omega_{a})\overline{\Gc M}_{dc}(\omega_{c})-\overline{\Gc M}_{da}(\omega_{a})\overline{\Gc M}_{bc}(\omega_{c})\Big]\sign(\mc{C})\Bigg\rangle_{\mathrm{MC}},
\end{align}
\begin{align}
M_{a\bar{b}\bar{c}d}^{3,ph}(\omega_{a},\omega_{\bar{b}})=\frac{1}{\langle\sign(\mc{C})\rangle_{\mathrm{MC}}}\Bigg\langle\Big[\bar{M}_{\bar{b}a}(\omega_{\bar{b}},\omega_{a})\overline{\Gc M\Gc}_{c\bar{d}}-\overline{\Gc M}_{ca}(\omega_{a})\overline{M\Gc}_{\bar{b}\bar{d}}(\omega_{\bar{b}})\Big]\sign(\mc{C})\Bigg\rangle_{\mathrm{MC}}.
\end{align}


\paragraph{Constructing $\chi^{3,r}$ from $M^{3,r}$}

\mbox{%
%
} \\[1.5ex] With 
\begin{align}
M_{abcd}^{3,pp,conn}(\omega_{a},\omega_{c})=M_{abcd}^{3,pp}(\omega_{a},\omega_{c})-\overline{\Gc M}_{ba}(\omega_{a})\overline{\Gc M}_{dc}(\omega_{c})+\overline{\Gc M}_{da}(\omega_{a})\overline{\Gc M}_{bc}(\omega_{c}),
\end{align}
\begin{align}
M_{a\bar{b}\bar{c}d}^{3,ph,conn}(\omega_{a},\omega_{\bar{b}})=M_{a\bar{b}\bar{c}d}^{3,ph}(\omega_{a},\omega_{\bar{b}})-\beta\delta_{\omega_{a},\omega_{\bar{b}}}\bar{M}_{\bar{b}a}(\omega_{a})\overline{\Gc M\Gc}_{c\bar{d}}+\overline{\Gc M}_{ca}(\omega_{a})\overline{M\Gc}_{\bar{b}\bar{d}}(\omega_{\bar{b}}),
\end{align}
we have that 
\begin{align}
\chi_{\bar{a}b\bar{c}d}^{3,pp,conn}(\omega_{a},\omega_{c})=\Gc_{\bar{a}i}(\omega_{a})\Gc_{\bar{c}j}(\omega_{c})M_{ibjd}^{3,pp,conn}(\omega_{a},\omega_{c}),
\end{align}
\begin{align}
\chi_{\bar{a}b\bar{c}d}^{3,ph,conn}(\omega_{\bar{a}},\omega_{b})=\Gc_{i\bar{a}}(\omega_{\bar{a}})\Gc_{b\bar{j}}(\omega_{b})M_{i\bar{j}\bar{c}d}^{3,ph,conn}(\omega_{\bar{a}},\omega_{b}).
\end{align}
Note that $M_{abcd}^{3,pp}(\omega_{a},\omega_{c})$ and $M_{a\bar{b}\bar{c}d}^{3,ph}(\omega_{a},\omega_{\bar{b}})$
can be either measured directly in frequencies, or calculated via
a Fourier transforms

\begin{align}
M_{abcd}^{3,pp}(\omega_{a},\omega_{c})=\int_{\tau_{a},\tau_{c}}e^{-i\tau_{a}\omega_{a}}e^{-i\tau_{c}\omega_{c}}M_{abcd}^{3,pp}(\tau_{a},\tau_{c})
\end{align}
\begin{align}
M_{a\bar{b}\bar{c}d}^{3,ph}(\omega_{a},\omega_{\bar{b}})=\int_{\tau_{a},\tau_{\bar{b}}}e^{-i\tau_{a}\omega_{a}}e^{i\tau_{\bar{b}}\omega_{\bar{b}}}M_{a\bar{b}\bar{c}d}^{3,ph}(\tau_{a},\tau_{\bar{b}})
\end{align}


\paragraph{Mixed fermion-boson notation}

Instead of using the frequencies of the fermionic operators, it is
often convenient to work in a mixed fermion-boson notation. It is
defined as 
\begin{align}
\tilde{M}_{abcd}^{3,pp}(\omega,\Omega)=M_{abcd}^{3,pp}(\omega,\Omega-\omega) & =\int_{\tau_{a},\tau_{c}}e^{i(\tau_{c}-\tau_{a})\omega}e^{-i\tau_{c}\Omega}M_{abcd}^{3,pp}(\tau_{a},\tau_{c})\\
 & =\int_{-\beta}^{0}d\tau'\int_{0}^{\beta}d\tau_{a}e^{-i(\tau'+\tau_{a})\omega}e^{i\tau'\Omega}M_{abcd}^{3,pp}(\tau_{a},-\tau')\\
 & =-\int_{0}^{\beta}d\tau'\int_{0}^{\beta}d\tau_{a}e^{-i(\tau'+\tau_{a})\omega}e^{i\tau'\Omega}M_{abcd}^{3,pp}(\tau_{a},\beta-\tau')\\
 & =\int_{0}^{\beta}d\tau'\int_{-\tau'}^{-\beta-\tau'}d\tau e^{i\tau\omega}e^{i\tau'\Omega}M_{abcd}^{3,pp}(-\tau-\tau',\beta-\tau')\\
 & =\int_{0}^{\beta}d\tau'\int_{-\tau'}^{\beta-\tau'}d\tau e^{i\tau\omega}e^{i\tau'\Omega}M_{abcd}^{3,pp}(\beta-\tau-\tau',\beta-\tau')\\
\end{align}
\begin{align}
\tilde{M}_{a\bar{b}\bar{c}d}^{3,ph}(\omega,\Omega)=M_{a\bar{b}\bar{c}d}^{3,ph}(\omega,\Omega+\omega)=\int_{\tau_{a},\tau_{\bar{b}}}e^{i(\tau_{\bar{b}}-\tau_{a})\omega}e^{i\tau_{\bar{b}}\Omega}M_{a\bar{b}\bar{c}d}^{3,ph}(\tau_{a},\tau_{\bar{b}}),
\end{align}
and accordingly for $\chi^{3,r}$. 

\subsection{Equal-time correlation functions $\chi^{2}$\label{sec:Measuring-M2}}

We now consider the case that two pairs of times are equal. In this
case we measure in imaginary time, and a Fourier transform is performed
in a post-processing step. We require again an intermediate object
\begin{equation}
\overline{\Gc M\Gc}_{\beta\bar{\alpha}}\equiv\overline{\Gc M\Gc}_{b\bar{a}}(\tau_{b},\tau_{a})\equiv\sum_{ij}\Gc_{b\bar{c}_{j}}(\tau_{b}-\tau_{j})[\hat{\Gc}^{-1}]_{ji}\Gc_{c_{i}\bar{a}}(\tau_{i}-\tau_{\bar{a}}).
\end{equation}
The measurements then read 
\begin{align}
M_{\bar{a}b\bar{c}d}^{2,pp}(\tau)=\frac{1}{\langle\sign(\mc{C})\rangle_{\mathrm{MC}}}\Bigg\langle\Big[\overline{\Gc M\Gc}_{b\bar{a}}(0^{+},\tau^{+})\overline{\Gc M\Gc}_{d\bar{c}}(0,\tau)-\overline{\Gc M\Gc}_{d\bar{a}}(0,\tau^{+})\overline{\Gc M\Gc}_{b\bar{c}}(0^{+},\tau)\Big]\sign(\mc{C})\Bigg\rangle_{\mathrm{MC}},
\end{align}
\begin{align}
M_{\bar{a}b\bar{c}d}^{2,ph}(\tau)=\frac{1}{\langle\sign(\mc{C})\rangle_{\mathrm{MC}}}\Bigg\langle\Big[\overline{\Gc M\Gc}_{b\bar{a}}(\tau,\tau^{+})\overline{\Gc M\Gc}_{d\bar{c}}(0,0^{+})-\overline{\Gc M\Gc}_{d\bar{a}}(0,\tau^{+})\overline{\Gc M\Gc}_{b\bar{c}}(\tau,0^{+})\Big]\sign(\mc{C})\Bigg\rangle_{\mathrm{MC}}.
\end{align}


\paragraph{Constructing $\chi^{2,r}$ from $M^{2,r}$}

\mbox{%
%
} \\[1.5ex] We have that 
\begin{align}
\chi_{\bar{a}b\bar{c}d}^{2,pp,conn}(\tau)=M_{\bar{a}b\bar{c}d}^{2,pp}(\tau)-\overline{\Gc M\Gc}_{b\bar{a}}(\beta-\tau)\overline{\Gc M\Gc}_{d\bar{c}}(\beta-\tau)+\overline{\Gc M\Gc}_{d\bar{a}}(\beta-\tau)\overline{\Gc M\Gc}_{b\bar{c}}(\beta-\tau),
\end{align}
\begin{align}
\chi_{\bar{a}b\bar{c}d}^{2,ph,conn}(\tau)=M_{\bar{a}b\bar{c}d}^{2,ph}(\tau)-\overline{\Gc M\Gc}_{b\bar{a}}(\beta^{-})\overline{\Gc M\Gc}_{d\bar{c}}(\beta^{-})-\overline{\Gc M\Gc}_{d\bar{a}}(\beta-\tau)\overline{\Gc M\Gc}_{b\bar{c}}(\tau).
\end{align}

\newpage
\comment{From here on, only old notes!}

\subsection{Two-point correlation function $\chi^{2}$ \label{sec:Measuring-the-two-point}}

The two-point correlation function is measured by operator insertion.
Let us perform the following interaction expansion: 
\begin{eqnarray}
\chi^{2} & = & \frac{1}{Z}\int{\cal D}[c^{\dagger},c]\,e^{c_{u}^{\dagger}[{\cal G}^{-1}]_{uv}c_{v}}\sum_{k,\mathbf{a},\mathbf{b}}(-)^{k}A_{k}[{\cal U},\mathbf{a},\mathbf{b}]n_{u}n_{v}c_{a_{1}}^{\dagger}...c_{a_{2k}}^{\dagger}c_{b_{1}}...c_{b_{2k}}\nonumber \\
 & = & \frac{Z_{0}}{Z}\sum_{k,\mathbf{a},\mathbf{b}}A_{k}[{\cal U},\mathbf{a},\mathbf{b}]\det D_{k}^{[u,v]}[{\cal G},\mathbf{a},\mathbf{b}]
\end{eqnarray}
where 
\begin{eqnarray}
D_{k}^{[u,v]}[{\cal G},\mathbf{a},\mathbf{b}]=\left(\begin{array}{ccccc}
{\cal G}_{uu} & {\cal G}_{uv} & {\cal G}_{ub_{1}} & ... & {\cal G}_{ub_{2k}}\\
{\cal G}_{vu} & {\cal G}_{vv} & {\cal G}_{vb_{1}} & ... & {\cal G}_{vb_{2k}}\\
{\cal G}_{a_{1}u} & {\cal G}_{a_{1}v} & {\cal G}_{a_{1}b_{1}} & ... & {\cal G}_{a_{1}b_{2k}}\\
... & ... & ... & ... & ...\\
{\cal G}_{a_{2k}u} & {\cal G}_{a_{2k}v} & ... & ... & {\cal G}_{a_{2k}b_{2k}}
\end{array}\right)
\end{eqnarray}
The determinant of this matrix can be expressed as 
\begin{eqnarray}
\det D_{k}^{[u,v]}=P\det D_{k}
\end{eqnarray}
where $D_{k}$ is defined in Eq.~\ref{defD}, while the determinant
ratio $P$ can be evaluated by the Woodbury identity and is what is
returned by try-insert-two. The effort is $O(k^{2})$ + the determinant
of $2\times2$ matrix which is negligible. In The case when $u$ and
$v$ are in different blocks, then, only one insertion is necessary
and we have for example 
\begin{equation}
\det D_{k}^{[u,v]}=\det D_{\sigma_{u}k}^{[u]}\det D_{\sigma_{v}k}^{[v]}=P_{\sigma_{v}}P_{\sigma_{u}}\det D_{\sigma_{v}k}\det D_{\sigma_{u}k}
\end{equation}

Finally, we use $\det D$ as the weight and the measurement is nothing
but 
\begin{equation}
\chi^{2}=\frac{1}{\langle\sign{\cal C}\rangle_{\mathrm{MC}}}\left\langle P\right\rangle _{\mathrm{MC}}
\end{equation}

The same can be done for the 3-point correlation function $\chi_{uvw}^{3}$
and for the 4-point correlation function $\chi_{uvwx}^{4}$ (both
defined in the vertex conventions document). However, the computational
effort for a single measurement here is $O(k^{2})$ while by cutting
lines is $O(1)$, because the determinant ratio is only one element
of the inverse matrix. However, the quality of measurement is better
controlled by doing the insertion. It becomes especially important
when we only have non-density density interactions -- in that case
measurement of correlators including density operators is not possible
by cutting lines in $\Xi$.

\subsection{Three-point correlation function $\chi^{3}$}

\paragraph{Computation from $\chi^{4}$}

\mbox{%
%
} \\[1.5ex] $M^{4,\mathrm{conn}}$, which is similar to the fully
reducible vertex $F$ (see section \ref{par:Fully-reducible-vertex}),
does not decay at large frequencies. See Figs\ref{fig:M4conn} and
\ref{fig:M4irred}.

Let us indeed look at the asymptotics of the summand: 
\[
\sum_{\omega'}\chi_{ijkk}^{4,\mathrm{conn}}(i\omega,i\omega',i\Omega)=\sum_{\omega'}{\cal G}_{u\bar{k}}(\omega'){\cal G}_{k\bar{v}}(\omega'+\Omega)\underbrace{M_{uvxy}^{4,\mathrm{conn}}(\omega',\omega,\Omega)}_{\rightarrow_{\omega'\rightarrow\infty}\mathrm{const}\neq0}{\cal G}_{x\bar{j}}(\omega+\Omega){\cal G}_{i\bar{y}}(\omega)
\]

A possible way out:

\begin{align}
\sum_{\omega'}\chi_{ijkk}^{4,\mathrm{conn}}(i\omega,i\omega',i\Omega) & =\sum_{\omega'}{\cal G}_{u\bar{k}}(\omega'){\cal G}_{k\bar{v}}(\omega'+\Omega)\underbrace{\left\{ M_{uvxy}^{4,\mathrm{conn}}(\omega',\omega,\Omega)-M_{uvxy}^{4,\mathrm{conn}}(\infty,\omega,\Omega)\right\} }_{\equiv M^{4,\mathrm{irred}}}{\cal G}_{x\bar{j}}(\omega+\Omega){\cal G}_{i\bar{y}}(\omega)\nonumber \\
 & \;\;+\underbrace{\sum_{\omega'}{\cal G}_{u\bar{k}}(\omega'){\cal G}_{k\bar{v}}(\omega'+\Omega)}_{\equiv P_{u\bar{k}k\bar{v}}(\Omega)}\left\{ M_{uvxy}^{4,\mathrm{conn}}(\infty,\omega,\Omega)\right\} {\cal G}_{x\bar{j}}(\omega+\Omega){\cal G}_{i\bar{y}}(\omega)\label{eq:sum_chi4conn}
\end{align}

and compute $P(\Omega)$ as a bubble in imaginary time. I have implemented
this, still issues, %
\begin{lyxgreyedout}
which probably come from the fact that, formally, $M_{uvxy}^{4,\mathrm{conn}}(\infty,\omega,\Omega)$
is not really defined%
\end{lyxgreyedout}
{} (?) (see Nils's paper, the asymptotics of $F$ are not really
what one should work with... should look at $\Phi^{r}$ in their respective
``native'' notation and extract $\mc{K}^{1}$, $\mc{K}^{2}$...?)

\begin{figure}
\begin{centering}
\includegraphics[width=0.8\columnwidth]{figures/M4conn} 
\par\end{centering}
\caption{$M^{4,\mathrm{conn}}(\omega,\omega',\Omega)$ ($U=1$)\label{fig:M4conn}}
\end{figure}

\begin{figure}
\begin{centering}
\includegraphics[width=0.8\columnwidth]{figures/M4irred} 
\par\end{centering}
\caption{$M^{4,\mathrm{irred}}(\omega,\omega',\Omega)$ ($U=1$)\label{fig:M4irred}
(defined in (\ref{eq:sum_chi4conn}))}
\end{figure}

An alternative: directly perform the inverse FT of $M^{4,\mathrm{conn}}$
(with one-variable tail fitting), and select $\tau'=0$. But will
it solve our problem?

\paragraph{Measurement by operator insertion}

\mbox{%
%
} \\[1.5ex] See section \ref{sec:Measuring-the-two-point}. To be
continued.

\pagebreak{}

\section{Implementation notes}

\subsection{General structure of the code}

To be continued

\subsubsection{Structure of the blocks}

\subsubsection{Structure of the interaction vertex}

\subsection{Design choices}

Important design choices...

\subsection{Monte-Carlo updates (moves)}

We have a list of vertices with all indices specified in advance,
including the auxiliary spin $s$. So when a vertex is chosen for
insertion one only needs to pick a vertex from a list and generate
random time(s).

\subsubsection{Move insert vertex }
\begin{itemize}
\item choose the vertex from the list 
\item if static, chose one time $\tau\in[0,\beta]$, otherwise two times
$\tau,\tau'$ 
\item add to the list of vertices in configuration 
\end{itemize}
Depending on whether the vertex is static or dynamic, the probability
to choose the given vertex and the corresponding times is 
\begin{eqnarray*}
P^{i\rightarrow f} & = & \frac{1}{N_{ver}\beta^{(2)}}\\
\end{eqnarray*}
Note that the $N_{s}$ is already included in $N_{ver}$, because
density-density vertices appear in $N_{s}$ copies with each possible
$s$. To go back, one just needs to choose this one out of all the
vertices present in the configuration, which there will be $k+1$
when we insert the chosen vertex, at the present (total) perturbation
order $k$ 
\begin{eqnarray*}
P^{f\rightarrow i} & = & \frac{1}{k+1}
\end{eqnarray*}
The weight ratio is 
\[
w=-\frac{N_{ver}\beta^{(2)}{\cal U}_{uv}}{k+1}\times\frac{\det\hat{{\cal G}}^{[u,v]}}{\det\hat{{\cal G}}}
\]
where ${\cal U}_{uv}$ is the interaction amplitude. For example if
the vertex is dynamic density-density, it is ${\cal U}_{uv}={\cal D}_{uv}/4$.
{\color{red} Why am I not missing here $1/N_{s}$ in the amplitude?
} Note also the minus sign in front which comes from $(-)^{k}$.
However, an extra minus sigh should be put in ${\cal J}^{\perp}$
because the sign does NOT depend on the perturbation order in ${\cal J}^{\perp}$.
Note that in the code one does not see $k+1$ but only $k$ because
at that point, the perturbation order is already increased (the vertex
is already added to the list of vertices in the configuration, and
if the move is not acepted will be removed from it). If accepted: 
\begin{itemize}
\item complete insertions in determinants 
\item together with the amliptude ${\cal U}_{uv}$ to be used when vertex
is evenutally removed. 
\end{itemize}
If rejected: 
\begin{itemize}
\item remove the vertex from the list of vertices present in the configuration 
\end{itemize}

\subsubsection{Move remove vertex}
\begin{itemize}
\item choose one vertex among the vertices present in the configuration
(double removal if same block, remove in different determinants if
different blocks) 
\end{itemize}
If accepted 
\begin{itemize}
\item complete operations on determinants 
\item erase vertex from the list of vertices in the configuration 
\end{itemize}
The weight ratio is 
\[
w=-\frac{k}{N_{ver}\beta^{(2)}{\cal U}_{uv}}\times\frac{\det\hat{{\cal G}}^{(u,v)}}{\det D}
\]


\subsection{Monte-Carlo measures}

The measures can be found in the folder c++/measures. One measure
is performed by class. The list of all measures is: 
\begin{enumerate}
\item measure\_chi3pmt ($\chi_{\pm}^{2}[\sigma,\sigma'](\tau)$) 
\item measure\_ft ($F_{ab}[\sigma](\tau)$) 
\item measure\_gw ($G_{ab}[\sigma](i\omega)$) 
\item measure\_M4t ($M_{abcd}^{4}[\sigma,\sigma'](\tau,\tau',\tau'')$) 
\item measure\_Mt ($M_{ab}[\sigma](\tau)$) 
\item measure\_nnt ($\chi_{ab}^{2}[\sigma,\sigma'](\tau)$) 
\item measure\_chipmt 
\item measure\_g2t ($\chi_{abc}^{3}[\sigma,\sigma'](\tau,\tau')$) 
\item measure\_hist: measure of the histogram of expansion oder. 
\item measure\_M4w 
\item measure\_nn ($\chi_{ab}^{2}[\sigma,\sigma']$) 
\item measure\_sign: measure of the Monte-Carlo sign 
\end{enumerate}
\pagebreak{}

\section{Benchmarks}

\subsection{Analytical limits}

\subsection{Benchmark against other codes}

\subsubsection{Comparison with CT-HYB}

\pagebreak{}

\section{Future development and known issues}

\subsection{Future development}

\subsection{Known issues}


\appendix
%dummy comment inserted by tex2lyx to ensure that this paragraph is not empty

\section{Useful identities}

\subsection{Results}

For later reference, let us remind the reader of the following identities
(see the next subsection for a derivation): 
\begin{align}
\frac{\partial f}{\partial G_{ab}} & =-G_{ua}^{-1}\frac{\partial f}{G_{uv}^{-1}}G_{bv}^{-1}\label{eq:df_dG_vs_df_dinvG}\\
\frac{\partial G_{ab}^{-1}}{\partial x} & =-G_{ac}^{-1}\frac{\partial G_{cd}}{\partial x}G_{db}^{-1}\label{eq:dinvG_dx}\\
\frac{\partial\mathrm{det}A}{\partial A_{ab}} & =A_{ba}^{-1}\mathrm{det}A\label{eq:ddet}
\end{align}

We also note that 
\[
Z_{0}=\mathrm{det}\Gc^{-1}
\]

Hence 
\begin{equation}
\frac{\partial Z_{0}}{\partial\Gc_{ab}^{-1}}=\Gc_{ba}Z_{0}\label{eq:dZ0_dinvG}
\end{equation}

and $\frac{\partial Z_{0}}{\partial\Gc_{ab}}=-\Gc_{ua}^{-1}\frac{\partial Z_{0}}{\partial\Gc_{uv}^{-1}}\Gc_{vb}^{-1}$
implies 
\begin{equation}
\frac{\partial Z_{0}}{\partial\Gc_{ab}}=-\Gc_{ba}^{-1}Z_{0}\label{eq:dZ0_dG}
\end{equation}


\subsection{Derivation}

\subsubsection{Derivative of inverse matrix\label{subsec:Derivative-of-inverse}}

\subsubsection{Derivative of determinant \label{subsec:Derivative-of-determinant}}

This can be proven the following way. We know the grassman gaussian
integral 
\begin{equation}
\int{\cal D}[c^{\dagger},c]e^{-c_{u}^{+}A_{uv}c_{v}}=\det\mathbf{A}
\end{equation}
So we can use this 
\begin{eqnarray}
\frac{\delta}{\delta A_{ij}}\det\mathbf{A} & = & \frac{\delta}{\delta A_{ij}}\int{\cal D}[c^{\dagger},c]e^{-c_{u}^{+}A_{uv}c_{v}}\\
 & = & \int{\cal D}[c^{\dagger},c](-c_{i}^{+}c_{j})e^{-c_{u}^{+}A_{uv}c_{v}}\\
 & = & -\det\mathbf{A}\langle c_{i}^{+}c_{j}\rangle
\end{eqnarray}
where the average is taken with respect to the action 
\[
S=c_{u}^{+}A_{uv}c_{v}
\]
and is by definition the bare propagator ${\cal G}_{ji}=\langle c_{i}^{+}c_{j}\rangle$.
However, we know that ${\cal G}=-\mathbf{A}^{-1}$ so 
\begin{eqnarray}
\frac{\delta}{\delta A_{ij}}\det\mathbf{A} & = & [\mathbf{A}^{-1}]_{ji}\,\det\mathbf{A}
\end{eqnarray}

\end{document}
