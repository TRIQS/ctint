%% LyX 2.2.1 created this file.  For more info, see http://www.lyx.org/.
%% Do not edit unless you really know what you are doing.
\documentclass[10pt]{article}
\usepackage[utf8]{inputenc}
\usepackage[a4paper]{geometry}
\geometry{verbose,tmargin=2cm,bmargin=2cm,lmargin=2cm,rmargin=2cm}
\setlength{\parskip}{\smallskipamount}
\setlength{\parindent}{0pt}
\usepackage{color}
\definecolor{note_fontcolor}{rgb}{1, 0, 0}
\usepackage{amsmath}
\usepackage{amssymb}
\usepackage{graphicx}
\usepackage[unicode=true,pdfusetitle,
 bookmarks=true,bookmarksnumbered=false,bookmarksopen=false,
 breaklinks=true,pdfborder={0 0 0},pdfborderstyle={},backref=false,colorlinks=true]
 {hyperref}

\makeatletter

%%%%%%%%%%%%%%%%%%%%%%%%%%%%%% LyX specific LaTeX commands.
%% The greyedout annotation environment
\newenvironment{lyxgreyedout}
  {\textcolor{note_fontcolor}\bgroup\ignorespaces}
  {\ignorespacesafterend\egroup}

%%%%%%%%%%%%%%%%%%%%%%%%%%%%%% User specified LaTeX commands.
\usepackage{color}


\newcommand{\up}{\uparrow}
\newcommand{\dn}{\downarrow}
\newcommand{\upup}{\uparrow\uparrow}
\newcommand{\updn}{\uparrow\downarrow}
\newcommand{\dnup}{\downarrow\uparrow}
\newcommand{\dndn}{\downarrow\downarrow}

%opening
\title{Cluster CTINT with dynamic interactions - implementation notes}

\makeatother

\begin{document}

\title{CTINT: derivation and implementation notes}
\maketitle
\begin{abstract}
In this document, we derive the equations of the CTINT algorithm,
and then give a detailed account of the implementation of this algorithm
using the TRIQS toolbox.
\end{abstract}
\tableofcontents{}

\pagebreak{}

\section{Derivation of the CTINT algorithm}

\subsection{General description of the algorithm}

\subsubsection{Definitions}

We consider the following impurity action 
\begin{equation}
S=S_{0}+S_{\mathrm{int}}
\end{equation}
with 
\begin{equation}
S_{0}=\sum_{ab}\iint\limits _{0}^{\beta}\mathrm{d}\tau\mathrm{d}\tau'\bar{c}_{a}(\tau)\left[-\mathcal{G}_{ab}^{-1}(\tau-\tau')\right]c_{b}(\tau')
\end{equation}

and
\begin{eqnarray}
\mathcal{S}_{\mathrm{int}} & = & \frac{1}{4N_{s}}\iint_{0}^{\beta}\mathrm{d}\tau\mathrm{d}\tau'\sum_{abs}\mathcal{D}_{ab}(\tau-\tau')\tilde{n}_{as}(\tau)\tilde{n}_{bs}(\tau')\\
 &  & +\frac{1}{2}\iint_{0}^{\beta}\mathrm{d}\tau\mathrm{d}\tau'\sum_{ij}{\cal J}_{ij}^{\perp}(\tau-\tau')s_{i}^{+}(\tau)s_{j}^{-}(\tau')
\end{eqnarray}
where 
\begin{eqnarray}
\mathcal{D}_{ab}(\tau) & = & \mathcal{U}_{ab}(\tau)+{\cal J}^{z}(\tau)\\
\tilde{n}_{as}(\tau) & \equiv & \hat{n}_{\sigma}(\tau)-\alpha^{sa}\\
\alpha^{sa} & \equiv & A_{a}+(2f_{sa}-1)\delta
\end{eqnarray}
and 
\[
s_{i}^{+}(\tau)=\bar{d}_{i\uparrow\tau}c_{i\downarrow\tau},\;\;\;s_{i}^{-}(\tau)=\bar{c}_{i\downarrow\tau}c_{i\uparrow\tau}
\]
$a$ and $b$ is combined orbital/spin index, $i$ is just orbital,
$\sigma$ just spin. Note however that $s^{+}$ can only act on a
singly occupied site with spin down fermion and vice versa for $s^{-}$.

Notes:
\begin{itemize}
\item the $1/N_{s}$ term is necessary to avoid multiple counting of the
interaction terms. However there is no sum over $s$ for the ${\cal J}^{\perp}$
interaction, and it does not figure in the renormalization of the
chemical potential. 
\item the prefactor $1/4$ avoids double counting due to two symmetries
$n_{i}n_{j}=n_{j}n_{i}$ and ${\cal D}(\tau)={\cal D}(-\tau)$. In
the case of spin-exchange interaction, $s_{i}^{+}s_{j}^{-}\neq s_{j}^{+}s_{i}^{-}$
but ${\cal J}^{\perp}(\tau)={\cal J}^{\perp}(-\tau)$ so we should
have $1/2$, but in the code I have $1/4$ because for each $(i,j)$
I add both $s_{i}^{+}s_{j}^{-}$ and $s_{i}^{-}s_{j}^{+}$ to the
list of vertices.
\item $\delta$ is a free small parameter which reduces the sign problem
and $f_{sa}$ is some function of auxiliary spin $s$ and combined
index $a$ such that it returns either $1$ or $0$. The $\alpha$'s
only appear in the interaction term. This comes at the cost of redefining
the noninteracting Green's function as 
\begin{equation}
\tilde{\mathcal{G}}_{ab}^{-1}(\tau)\equiv\mathcal{G}_{ab}^{-1}(\tau)-\frac{1}{N_{s}}\left[\sum_{cs}\alpha_{cs}\int_{0}^{\beta}\mathrm{d}\bar{\tau}\mathcal{D}_{ac}(\tau-\bar{\tau})\right]\delta_{ab}\delta(\tau)
\end{equation}
\end{itemize}
We can compute 
\begin{eqnarray*}
\int_{0}^{\beta}\mathrm{d}\bar{\tau}\mathcal{D}_{ab}(\tau-\bar{\tau}) & = & \int_{\tau-\beta}^{\tau}\mathrm{d}\tilde{\tau}\mathcal{D}_{ab}(\tilde{\tau})\\
 & = & \int_{\tau-\beta}^{0^{-}}\mathrm{d}\tilde{\tau}\mathcal{D}_{ab}(\tilde{\tau})+\int_{0^{+}}^{\tau}\mathrm{d}\tilde{\tau}\mathcal{D}_{ab}(\tilde{\tau})\\
 & = & K'_{ab}(0^{-})-K'_{ab}(\tau-\beta)+K'_{ab}(\tau)-K'_{ab}(0^{+})\\
 & = & -2K'_{ab}(0^{+})
\end{eqnarray*}
where $K(\tau)$ is defined as $K''(\tau)=\mathcal{D}(\tau)$ for
$\tau\in]0,\beta[$, and $K(0^{+})=K(\beta^{-})=0$.

\subsubsection{Interaction expansion}

The CT-INT algorithm consists in expanding the partition function
\begin{equation}
Z\equiv\int\mathcal{D}[\bar{c},c]e^{\bar{c}_{a}\mathcal{G}_{ab}^{-1}c_{b}-S_{\mathrm{int}}}\label{eq:partition_function_def}
\end{equation}
in powers of $\mathcal{S}_{\mathrm{int}}$. One obtains:

\begin{eqnarray*}
Z & = & Z_{0}\sum_{k=0}^{\infty}\frac{(-)^{k}}{k!\,4^{k}}\sum_{k_{\perp}=0}^{\infty}\frac{(-)^{k_{\perp}}}{k_{\perp}!\,2^{k_{\perp}}}\int_{0}^{\beta}\mathrm{d}\tau\ldots\mathrm{d}\tau\int\limits _{0}^{\beta}\mathrm{d}\tau'_{1}\ldots\mathrm{d}\tau'_{k+k_{\perp}}\times\\
 &  & \times\prod_{i=1}^{k}\frac{{\cal D}_{a_{i}b_{i}}(\tau_{i}-\tau_{i}')}{4N_{s}}\prod_{j=k+1}^{k+k_{\perp}}\frac{{\cal J}^{\perp}(\tau_{j}-\tau'_{j})}{2}\times\\
 &  & \times\sum_{\substack{a_{1}\dots a_{k}\\
b_{1}\dots b_{k}
}
}\sum_{\substack{c_{1}\dots c_{k_{\perp}}\\
d_{1}\dots d_{k_{\perp}}
}
}\Big\langle T\tilde{n}_{a_{1}}(\tau_{1})\dots\tilde{n}_{a_{k}}(\tau_{k})\tilde{n}_{b_{1}}(\tau_{1}')\dots\tilde{n}_{b_{k}}(\tau_{k}')\times\\
 &  & \times s_{c_{1}}^{\dagger}(\tau_{k+1})...s_{c_{k_{\perp}}}^{\dagger}(\tau_{k+k_{\perp}})s_{d_{1}}^{-}(\tau'_{k+1})...s_{d_{k_{\perp}}}^{-}(\tau'_{k+k_{\perp}})\Big\rangle_{0}
\end{eqnarray*}

where $T_{\tau}$ is the time ordering operator. We now order the
times of the first integrals, but for $k$ and $k_{\perp}$times independently
so that the prefactor factorials are cancelled. Using the properties
of the time ordering operator

\begin{eqnarray*}
Z & = & Z_{0}\sum_{k=0}^{\infty}\sum_{k_{\perp}=0}^{\infty}(-)^{k+k_{\perp}}\int_{>}\mathrm{d}\tau\ldots\mathrm{d}\tau\int_{>}\mathrm{d}\tau\ldots\mathrm{d}\tau\int\limits _{0}^{\beta}\mathrm{d}\tau'_{1}\ldots\mathrm{d}\tau'_{k+k_{\perp}}\\
 &  & \times\prod_{i=1}^{k}\frac{{\cal D}_{a_{i}b_{i}}(\tau_{i}-\tau_{i}')}{4N_{s}}\prod_{j=k+1}^{k+k_{\perp}}\frac{{\cal J}^{\perp}(\tau_{j}-\tau'_{j})}{2}\times\\
 &  & \times\sum_{\substack{a_{1}\dots a_{k}\\
b_{1}\dots b_{k}
}
}\sum_{\substack{c_{1}\dots c_{k_{\perp}}\\
d_{1}\dots d_{k_{\perp}}
}
}\Big\langle T\tilde{n}_{a_{1}}(\tau_{1})\dots\tilde{n}_{a_{k}}(\tau_{k})\tilde{n}_{b_{1}}(\tau_{1}')\dots\tilde{n}_{b_{k}}(\tau_{k}')\times\\
 &  & \times s_{c_{1}}^{\dagger}(\tau_{k+1})...s_{c_{k_{\perp}}}^{\dagger}(\tau_{k+k_{\perp}})s_{d_{1}}^{-}(\tau'_{k+1})...s_{d_{k_{\perp}}}^{-}(\tau'_{k+k_{\perp}})\Big\rangle_{0}
\end{eqnarray*}
Note that we do not order the integrals on $\tau'$.

Assuming no spin-offdiagonal terms in the bare Green's function, we
split the average in spin up and down parts. However, for this we
need to rearrange the $d$ fields which will produce an additional
minus sign in front 
\begin{eqnarray*}
Z & = & Z_{0}\sum_{k=0}^{\infty}\sum_{k_{\perp}=0}^{\infty}(-)^{k+k_{\perp}}\int_{>}\mathrm{d}\tau\ldots\mathrm{d}\tau\int_{>}\mathrm{d}\tau\ldots\mathrm{d}\tau\int\limits _{0}^{\beta}\mathrm{d}\tau'_{1}\ldots\mathrm{d}\tau'_{k+k_{\perp}}\times\\
 &  & \times\prod_{i=1}^{k}\frac{{\cal D}_{a_{i}b_{i}}(\tau_{i}-\tau_{i}')}{4N_{s}}\prod_{j=k+1}^{k+k_{\perp}}\frac{{\cal J}^{\perp}(\tau_{j}-\tau'_{j})}{2}\times\\
 &  & \times(-)^{k_{\perp}}\sum_{s_{1...k}}\sum_{\substack{i_{1}\dots i_{k+k_{\perp}}}
}\Big\langle T(\bar{c}_{i_{1}\uparrow\tau_{1}}c_{i_{1}\uparrow\tau_{1}}-\alpha^{s_{1}\uparrow})\dots(\bar{c}_{i_{k}\uparrow\tau_{k}}c_{i_{k}\uparrow\tau_{k}}-\alpha^{s_{k}\uparrow})\\
 &  & \;\;\;\;\;\;\;\;\;\;\;\;\;\bar{c}_{i_{k+1}\uparrow\tau_{k+1}}c_{i_{k+1}\uparrow\tau'_{k+1}}...\bar{c}_{i_{k+k_{\perp}}\uparrow\tau_{k+k_{\perp}}}c_{i_{k+k_{\perp}}\uparrow\tau'_{k+k_{\perp}}}\Big\rangle_{0}\times\\
 &  & \times\sum_{\substack{j_{1}\dots j_{k+k_{\perp}}}
}\Big\langle T(\bar{c}_{j_{1}\downarrow\tau_{1}}c_{j_{1}\downarrow\tau_{1}}-\alpha^{s_{1}\downarrow})\dots(\bar{c}_{j_{k}\downarrow\tau_{k}}c_{j_{k}\downarrow\tau_{k}}-\alpha^{s_{k}\downarrow})\\
 &  & \;\;\;\;\;\;\;\;\;\;\;\;\;c_{j_{k+1}\downarrow\tau_{k+1}}\bar{c}_{j_{k+1}\downarrow\tau'_{k+1}}...c_{j_{k+k_{\perp}}\downarrow\tau_{k+k_{\perp}}}\bar{c}_{j_{k+k_{\perp}}\downarrow\tau'_{k+k_{\perp}}}\Big\rangle_{0}
\end{eqnarray*}
such that we see that the sign does NOT depend on the perturbation
order in ${\cal J}^{\perp}$.

The expression above can be written in a more compact form: 
\begin{eqnarray*}
Z & = & Z_{0}\sum_{k\geq0,\mathbf{a},\mathbf{b}}A_{k}[{\cal U},\mathbf{a},\mathbf{b}]\langle c_{a_{1}}^{\dagger}...c_{a_{2k}}^{\dagger}c_{b_{1}}...c_{b_{2k}}\rangle_{0}
\end{eqnarray*}
where $\mathbf{a}\equiv\{a_{i}\}\equiv(a_{1},...,a_{2k})$, and analogously
for $\mathbf{b}$. We now apply Wick's theorem to evaluate the noninteracting
expectation value, getting 
\begin{eqnarray}
Z & = & Z_{0}\Xi\label{eq:Z_as_Z0Xi}
\end{eqnarray}

with
\begin{equation}
\Xi\equiv\sum_{k\geq0,\mathbf{a},\mathbf{b}}A_{k}[{\cal U},\mathbf{a},\mathbf{b}]\det\hat{\mathcal{G}}_{k}[{\cal G},\mathbf{a},\mathbf{b}]\label{eq:Xi_def}
\end{equation}
and we have defined the matrix $\hat{\mathcal{G}}$ as: 
\begin{eqnarray}
[\hat{\mathcal{G}}]_{ij}\equiv\mathcal{G}_{a_{i}b_{j}}\label{defD}
\end{eqnarray}

and the second term can be considered a functional of only the interaction
and the bare propagator 

\subsubsection{Definitions of the main correlation functions}

\paragraph{Single-particle correlation function}

The single-particle Green's function is defined as
\begin{equation}
G_{ab}\equiv-\langle c_{a}\bar{c}_{b}\rangle\label{eq:G_def}
\end{equation}

Using (\ref{eq:partition_function_def}), we note $\frac{\partial Z}{\partial\mathcal{G}_{ab}^{-1}}=Z\langle\bar{c}_{a}c_{b}\rangle$
i.e
\begin{equation}
G_{ab}=\frac{1}{Z}\frac{\partial Z}{\partial\mathcal{G}_{ba}^{-1}}\label{eq:G_as_derivative_of_Z}
\end{equation}


\paragraph{Four-point function}

The four-point function $\chi_{abcd}^{4}$ is defined as
\begin{equation}
\chi_{abcd}^{4}\equiv\langle\bar{c}_{a}c_{b}\bar{c}_{c}c_{d}\rangle\label{eq:chi4_def}
\end{equation}

Using (\ref{eq:partition_function_def}), we note $\frac{\partial Z}{\partial\mathcal{G}_{ab}^{-1}\partial\mathcal{G}_{cd}^{-1}}=Z\langle\bar{c}_{a}c_{b}\bar{c}_{c}c_{d}\rangle$
i.e
\begin{equation}
\chi_{abcd}^{4}=\frac{1}{Z}\frac{\partial Z}{\partial\mathcal{G}_{ab}^{-1}\partial\mathcal{G}_{cd}^{-1}}\label{eq:chi4_as_derivative_of_Z}
\end{equation}

We define its Fourier transform as:

\begin{equation}
\chi_{abcd}^{4}(i\omega,i\omega',i\Omega)\equiv\iiiint e^{-i\omega\tau_{1}}e^{(i\omega+i\Omega)\tau_{2}}e^{-(i\omega'+i\Omega)\tau_{3}}e^{i\omega'\tau_{4}}\chi_{abcd}^{4}(\tau_{1},\tau_{2},\tau_{3},\tau_{4})\label{eq:FT_4_point_fcts}
\end{equation}
which can be rewritten symbolically as:
\[
\chi_{abcd}^{4}(i\omega,i\omega',i\Omega)=\langle c_{a}^{\dagger}(i\omega)c_{b}(i\omega+i\nu)c_{c}^{\dagger}(i\omega'+i\nu)c_{d}(i\omega')\rangle
\]

The inverse Fourier transform is:
\begin{equation}
\chi_{abcd}^{4}(\tau_{1},\tau_{2},\tau_{3},\tau_{4})\equiv\sum_{i\omega,i\omega',i\Omega}e^{i\omega\tau_{1}}e^{-(i\omega+i\Omega)\tau_{2}}e^{(i\omega'+i\Omega)\tau_{3}}e^{-i\omega'\tau_{4}}\chi_{abcd}^{4}(i\omega,i\omega',i\Omega)\label{eq:inverse_FT_4_point_fcts}
\end{equation}


\paragraph{Alternative definition of $\chi^{4}$}

Let us define:
\begin{equation}
\tilde{\chi}_{a\bar{b}c\bar{d}}^{4}(\tau_{1},\tau_{2},\tau_{3},\tau_{4})\equiv\langle c_{a}(\tau_{1})\bar{c}_{\bar{b}}(\tau_{2})c_{c}(\tau_{3})\bar{c}_{\bar{d}}(\tau_{4})\rangle\label{eq:chi4_tilde_def}
\end{equation}

We have
\begin{equation}
\tilde{\chi}_{abcd}^{4}(\tau_{1},\tau_{2},\tau_{3},\tau_{4})=\chi_{badc}^{4}(\tau_{2},\tau_{1},\tau_{4},\tau_{3})\label{eq:chi4_chi4tilde_time}
\end{equation}

In frequency,

\begin{equation}
\tilde{\chi}_{abcd}^{4}(i\omega,i\omega',i\Omega)\equiv\iiiint e^{i\omega\tau_{1}}e^{-(i\omega+i\Omega)\tau_{2}}e^{(i\omega'+i\Omega)\tau_{3}}e^{-i\omega'\tau_{4}}\tilde{\chi}_{abcd}^{4}(\tau_{1},\tau_{2},\tau_{3},\tau_{4})\label{eq:FT_4_point_fcts-1}
\end{equation}

Hence:

\begin{align}
\chi_{abcd}^{4}(i\omega,i\omega',i\Omega) & \equiv\iiiint e^{-i\omega\tau_{1}}e^{(i\omega+i\Omega)\tau_{2}}e^{-(i\omega'+i\Omega)\tau_{3}}e^{i\omega'\tau_{4}}\tilde{\chi}_{badc}^{4}(\tau_{2},\tau_{1},\tau_{4},\tau_{3})\nonumber \\
 & =\iiiint e^{-i\omega\tau_{2}}e^{(i\omega+i\Omega)\tau_{1}}e^{-(i\omega'+i\Omega)\tau_{4}}e^{i\omega'\tau_{3}}\tilde{\chi}_{badc}^{4}(\tau_{1},\tau_{2},\tau_{3},\tau_{4})\nonumber \\
 & =\iiiint e^{(i\omega+i\Omega)\tau_{1}}e^{-i\omega\tau_{2}}e^{i\omega'\tau_{3}}e^{-(i\omega'+i\Omega)\tau_{4}}\tilde{\chi}_{badc}^{4}(\tau_{1},\tau_{2},\tau_{3},\tau_{4})\nonumber \\
 & =\tilde{\chi}_{badc}^{4}(\omega+\Omega,\omega'+\Omega,-\Omega)\label{eq:chi4_chi4tilde_freq}
\end{align}

or equivalently
\begin{equation}
\tilde{\chi}_{abcd}^{4}(i\omega,i\omega',i\Omega)=\chi_{badc}^{4}(\omega+\Omega,\omega'+\Omega,-\Omega)\label{eq:chi4tilde_chi4_freq}
\end{equation}

The connected correlation function is defined as:
\begin{equation}
\tilde{\chi}_{abcd}^{4,\mathrm{conn}}\equiv\tilde{\chi}_{abcd}^{4}-G_{ab}G_{cd}+G_{ad}G_{cb}\label{eq:chi4tilde_conn_def}
\end{equation}

In frequency space:

\begin{equation}
\tilde{\chi}_{abcd}^{4,\mathrm{conn}}(i\omega,i\omega',i\Omega)=\tilde{\chi}_{abcd}^{4}(i\omega,i\omega',i\Omega)-G_{ab}(\omega)G_{cd}(\omega')\beta\delta_{\Omega}+G_{ad}(\omega+\Omega)G_{cb}(\omega)\beta\delta_{\omega\omega'}\label{eq:chi4tilde_conn_freq}
\end{equation}

Last, we note that the fully reducible two-particle vertex is defined
as
\begin{equation}
F_{\bar{a}b\bar{c}d}\equiv G_{\bar{a}a}^{-1}G_{\bar{b}b}^{-1}\chi_{a\bar{b}c\bar{d}}^{4,\mathrm{conn}}G_{\bar{c}c}^{-1}G_{\bar{d}d}^{-1}\label{eq:F_def}
\end{equation}


\paragraph{Two-point correlation function}

We define

\begin{equation}
\chi_{ab}^{2}\equiv\langle n_{a}n_{b}\rangle\label{eq:chi3_def}
\end{equation}


\paragraph{Three-point function}

The three-point correlation function is defined as:
\begin{equation}
\chi_{a\bar{b}c}^{3}\equiv-\langle c_{a}\bar{c}_{\bar{b}}n_{c}\rangle\label{eq:chi3_def-1}
\end{equation}

Note the minus sign compared to the definition in TRILEX. Its Fourier
transform is defined (symbolically) as

\[
\chi_{aa'b}^{3}(i\omega,i\nu)\equiv-\langle c_{a}(i\omega)\bar{c}_{a'}(i\omega+i\nu)n_{b}(i\nu)\rangle
\]

The connected function is

\begin{equation}
\chi_{ijk}^{3,\mathrm{conn}}(\omega,\Omega)\equiv\chi_{ijk}^{3}(\omega,\Omega)+G_{ij}(\omega)n_{k}\beta\delta_{\Omega}-G_{ik}(\omega+\Omega)G_{kj}(\omega)\label{eq:chi3_conn_def}
\end{equation}


\subsubsection{Useful identities}

For later reference, let us remind the reader of the following identities
(see Appendix \ref{sec:Details} for a derivation):
\begin{align}
\frac{\partial f}{\partial G_{ab}} & =-G_{ua}^{-1}\frac{\partial f}{G_{uv}^{-1}}G_{bv}^{-1}\label{eq:df_dG_vs_df_dinvG}\\
\frac{\partial G_{ab}^{-1}}{\partial x} & =-G_{ac}^{-1}\frac{\partial G_{cd}}{\partial x}G_{db}^{-1}\label{eq:dinvG_dx}\\
\frac{\partial\mathrm{det}A}{\partial A_{ab}} & =A_{ba}^{-1}\mathrm{det}A\label{eq:ddet}
\end{align}

We also note that
\[
Z_{0}=\mathrm{det}\mathcal{G}^{-1}
\]

Hence
\begin{equation}
\frac{\partial Z_{0}}{\partial\mathcal{G}_{ab}^{-1}}=\mathcal{G}_{ba}Z_{0}\label{eq:dZ0_dinvG}
\end{equation}

and $\frac{\partial Z_{0}}{\partial\mathcal{G}_{ab}}=-\mathcal{G}_{ua}^{-1}\frac{\partial Z_{0}}{\partial\mathcal{G}_{uv}^{-1}}\mathcal{G}_{vb}^{-1}$
implies
\begin{equation}
\frac{\partial Z_{0}}{\partial\mathcal{G}_{ab}}=-\mathcal{G}_{ba}^{-1}Z_{0}\label{eq:dZ0_dG}
\end{equation}


\subsubsection{From $\tilde{\chi}^{4}$ to $\chi^{3}$}

Let us first note that:

\begin{align}
\chi_{ijk}^{3}(\omega,\Omega) & =-\iint e^{i\omega\tau+i\Omega\tau'}\langle Tc_{i}(\tau)\bar{c}_{j}(0)\bar{c}_{k}(\tau'^{+})c_{k}(\tau')\rangle=\iint e^{i\omega\tau+i\Omega\tau'}\langle Tc_{i}(\tau)\bar{c}_{j}(0)c_{k}(\tau')\bar{c}_{k}(\tau'^{+})\rangle\label{eq:chi3_interm}
\end{align}

Here, we have just used the properties of the time-ordering operator.
On the other hand:

\begin{align*}
\tilde{\chi}_{abcd}^{4}(i\omega,i\omega',i\Omega) & =\iiiint e^{i\omega(\tau_{1}-\tau_{2})}e^{i\Omega(\tau_{3}-\tau_{2})}e^{i\omega'(\tau_{3}-\tau_{4})}\tilde{\chi}_{abcd}^{4}(\tau_{1},\tau_{2},\tau_{3},\tau_{4})\\
 & =\iiint e^{i\omega(\tau_{1})}e^{i\Omega(\tau_{3})}e^{i\omega'(\tau_{3}-\tau_{4})}\tilde{\chi}_{abcd}^{4}(\tau_{1},0^{+},\tau_{3},\tau_{4})
\end{align*}

Summed over $\omega'$ (and multiplied by $e^{i\omega0^{+}}$):

\begin{align*}
\sum_{\omega'}\tilde{\chi}_{abcd}^{4}(i\omega,i\omega',i\Omega)e^{i\omega'0^{+}} & =\iiint e^{i\omega(\tau_{1})}e^{i\Omega(\tau_{3})}\sum_{\omega'}e^{i\omega'(\tau_{3}-\tau_{4})}e^{i\omega'0^{+}}\langle Tc_{a}(\tau_{1})\bar{c}_{b}(0^{+})c_{c}(\tau_{3})\bar{c}_{d}(\tau_{4})\rangle\\
 & =\iiint e^{i\omega(\tau_{1})}e^{i\Omega(\tau_{3})}\langle Tc_{a}(\tau_{1})\bar{c}_{b}(0^{+})c_{c}(\tau_{3})\bar{c}_{d}(\tau_{3}^{+})\rangle
\end{align*}

The $e^{i\omega0^{+}}$ factor is important to get the $\tau_{3}^{+}$.

One recognizes the term in Eq. (\ref{eq:chi3_interm}). Thus:

\begin{equation}
\chi_{ijk}^{3}(\omega,\Omega)=\sum_{\omega'}\tilde{\chi}_{ijkk}^{4}(i\omega,i\omega',i\Omega)e^{i\omega'0^{+}}\label{eq:chi3_chi4_rel}
\end{equation}
This sum on $\omega'$ is not well behaved because of the disconnected
terms. Let us remove them: replacing $\tilde{\chi}^{4}$ with its
connected part using (\ref{eq:chi4tilde_conn_freq}), we obtain:

\begin{align}
\chi_{ijk}^{3}(\omega,\Omega) & =\sum_{\omega'}\tilde{\chi}_{ijkk}^{4,\mathrm{conn}}(i\omega,i\omega',i\Omega)e^{i\omega'0^{+}}-G_{ij}(\omega)\sum_{\omega'}G_{kk}(\omega')e^{i\omega'0^{+}}\beta\delta_{\Omega}+G_{ik}(\omega+\Omega)G_{kj}(\omega)\nonumber \\
 & =\sum_{\omega'}\tilde{\chi}_{ijkk}^{4,\mathrm{conn}}(i\omega,i\omega',i\Omega)e^{i\omega'0^{+}}-G_{ij}(\omega)n_{k}\beta\delta_{\Omega}+G_{ik}(\omega+\Omega)G_{kj}(\omega)\label{eq:chi3_vs_chi4tilde_conn_final}
\end{align}

Hence, using (\ref{eq:chi3_conn_def}):

\begin{align}
\chi_{ijk}^{3,\mathrm{conn}}(\omega,\Omega) & =\sum_{\omega'}\tilde{\chi}_{ijkk}^{4,\mathrm{conn}}(i\omega,i\omega',i\Omega)e^{i\omega'0^{+}}\label{eq:chi3conn_vs_chi4tilde_conn_final}
\end{align}

This result makes sense.\footnote{Compare (\ref{eq:chi3_chi4_rel}) to 

\begin{equation}
\chi_{aa'b}^{3}(i\omega,i\nu)=\frac{1}{2}G_{aa'}(i\omega)\beta\delta_{\nu}+\frac{1}{\beta}\sum_{i\omega'}\tilde{\chi}_{aa'bb}^{4}(i\omega,i\omega',i\nu)\label{eq:chi3_from_chi4}
\end{equation}

and my previous derivation:

tentative proof:
\[
\tilde{\chi}^{4}(i\omega,i\omega',i\Omega)\equiv\langle c_{\omega}c_{\omega+\Omega}^{\dagger}c_{\omega'+\Omega}c_{\omega'}^{\dagger}\rangle
\]

Let us sum over $\omega'$:
\begin{align}
\frac{1}{\beta}\sum_{\omega'}\tilde{\chi}^{4}(i\omega,i\omega',i\Omega) & =\iiiint_{\tau_{1}\tau_{2}\tau_{3}\tau_{4}}e^{i\omega\tau_{1}-i(\omega+\Omega)\tau_{2}+\Omega\tau_{3}}\underbrace{\frac{1}{\beta}\sum_{\omega'}e^{i\omega'(\tau_{3}-\tau_{4})}}_{=\frac{1}{2}\delta_{\tau_{3}-\tau_{4}}}\langle c_{\tau_{1}}c_{\tau_{2}}^{\dagger}c_{\tau_{3}}c_{\tau_{4}}^{\dagger}\rangle\\
 & =\frac{1}{2}\iiint_{\tau_{1}\tau_{2}\tau_{3}}e^{i\omega\tau_{1}-i(\omega+\Omega)\tau_{2}+\Omega\tau_{3}}\langle c_{\tau_{1}}c_{\tau_{2}}^{\dagger}\left(1-c_{\tau_{3}}^{\dagger}c_{\tau_{3}}\right)\rangle\\
 & =\frac{1}{2}\iiint_{\tau_{1}\tau_{2}\tau_{3}}e^{i\omega\tau_{1}-i(\omega+\Omega)\tau_{2}+\Omega\tau_{3}}\langle c_{\tau_{1}}c_{\tau_{2}}^{\dagger}\rangle-\frac{1}{2}\iiint_{\tau_{1}\tau_{2}\tau_{3}}e^{i\omega\tau_{1}-i(\omega+\Omega)\tau_{2}+\Omega\tau_{3}}\langle c_{\tau_{1}}c_{\tau_{2}}^{\dagger}c_{\tau_{3}}^{\dagger}c_{\tau_{3}}\rangle\\
 & =-\frac{1}{2}G_{\omega}\beta\delta_{i\Omega}-\frac{1}{2}\langle c_{\omega}c_{\omega+\Omega}^{\dagger}n_{\Omega}\rangle
\end{align}

Hence:
\begin{equation}
\chi_{\omega\Omega}^{3}\equiv-\langle c_{\omega}c_{\omega+\Omega}^{\dagger}n_{\Omega}\rangle=G_{\omega}\beta\delta_{\Omega}+\frac{2}{\beta}\sum_{\omega'}\tilde{\chi}^{4}(i\omega,i\omega',i\Omega)\label{eq:chi3_chi4}
\end{equation}

which is different from Eq. (\ref{eq:chi3_from_chi4}).} One can expect that the sum on $\omega'$ of $\tilde{\chi}^{4,\mathrm{conn}}$
is better behaved than that of $\tilde{\chi}^{4}$. %
\begin{lyxgreyedout}
In fact, it is not the case (I obtain the same result as direct sum
of $\tilde{\chi}^{4}$). Reason(?): $M^{4,\mathrm{conn}}$, which
is similar to the fully reducible vertex $F$ (see section \ref{par:Fully-reducible-vertex}),
does not decay at large frequencies. See Figs\ref{fig:M4conn} and
\ref{fig:M4irred}. %
\end{lyxgreyedout}
Let us indeed look at the asymptotics of the summand: 
\[
\sum_{\omega'}\tilde{\chi}_{ijkk}^{4,\mathrm{conn}}(i\omega,i\omega',i\Omega)=\sum_{\omega'}{\cal G}_{u\bar{k}}(\omega'){\cal G}_{k\bar{v}}(\omega'+\Omega)\underbrace{M_{uvxy}^{4,\mathrm{conn}}(\omega',\omega,\Omega)}_{\rightarrow_{\omega'\rightarrow\infty}\mathrm{const}\neq0}{\cal G}_{x\bar{j}}(\omega+\Omega){\cal G}_{i\bar{y}}(\omega)
\]

A possible way out:

\begin{align}
\sum_{\omega'}\tilde{\chi}_{ijkk}^{4,\mathrm{conn}}(i\omega,i\omega',i\Omega) & =\sum_{\omega'}{\cal G}_{u\bar{k}}(\omega'){\cal G}_{k\bar{v}}(\omega'+\Omega)\underbrace{\left\{ M_{uvxy}^{4,\mathrm{conn}}(\omega',\omega,\Omega)-M_{uvxy}^{4,\mathrm{conn}}(\infty,\omega,\Omega)\right\} }_{\equiv M^{4,\mathrm{irred}}}{\cal G}_{x\bar{j}}(\omega+\Omega){\cal G}_{i\bar{y}}(\omega)\nonumber \\
 & \;\;+\underbrace{\sum_{\omega'}{\cal G}_{u\bar{k}}(\omega'){\cal G}_{k\bar{v}}(\omega'+\Omega)}_{\equiv P_{u\bar{k}k\bar{v}}(\Omega)}\left\{ M_{uvxy}^{4,\mathrm{conn}}(\infty,\omega,\Omega)\right\} {\cal G}_{x\bar{j}}(\omega+\Omega){\cal G}_{i\bar{y}}(\omega)\label{eq:sum_chi4conn}
\end{align}

and compute $P(\Omega)$ as a bubble in imaginary time. I have implemented
this, still issues, %
\begin{lyxgreyedout}
which probably come from the fact that, formally, $M_{uvxy}^{4,\mathrm{conn}}(\infty,\omega,\Omega)$
is not really defined%
\end{lyxgreyedout}
{} (?) (see Nils's paper, the asymptotics of $F$ are not really what
one should work with... should look at $\Phi^{r}$ in their respective
``native'' notation and extract $\mathcal{K}^{1}$, $\mathcal{K}^{2}$...?)

\begin{figure}
\begin{centering}
\includegraphics[width=0.8\columnwidth]{/home/tayral/Documents/2016/saclay/ctint/figures/M4conn}
\par\end{centering}
\caption{$M^{4,\mathrm{conn}}(\omega,\omega',\Omega)$ ($U=1$)\label{fig:M4conn}}

\end{figure}

\begin{figure}
\begin{centering}
\includegraphics[width=0.8\columnwidth]{/home/tayral/Documents/2016/saclay/ctint/figures/M4irred}
\par\end{centering}
\caption{$M^{4,\mathrm{irred}}(\omega,\omega',\Omega)$ ($U=1$)\label{fig:M4irred}
(defined in (\ref{eq:sum_chi4conn}))}
\end{figure}

An alternative: directly perform the inverse FT of $M^{4,\mathrm{conn}}$
(with one-variable tail fitting), and select $\tau'=0$. But will
it solve our problem?

\subsection{Monte-Carlo algorithm}

The sum in Eq. (\ref{eq:Xi_def}) is sampled using the Metropolis-Hastings
algorithm, i.e we write it as a sum over configurations
\[
\Xi=\sum_{\mathcal{C}}w_{\mathcal{C}}
\]

The expectation value of a observable $\hat{O}$ is computed as:
\begin{equation}
\langle\hat{O}\rangle\approx\langle\hat{O}\rangle_{\mathrm{MC}}\equiv\frac{\sum_{\mathcal{C}}w_{\mathcal{C}}O_{\mathcal{C}}}{\sum_{\mathcal{C}}w_{\mathcal{C}}}=\frac{1}{\Xi}\sum_{\mathcal{C}}w_{\mathcal{C}}O_{\mathcal{C}}\label{eq:MC_average_def}
\end{equation}


\subsubsection{Single-particle Green's function $G$}

Starting from Eqs (\ref{eq:G_as_derivative_of_Z}) and (\ref{eq:Z_as_Z0Xi}-\ref{eq:Xi_def}),
we obtain
\begin{align}
G_{ab} & =\frac{1}{Z}\left[\frac{\partial Z_{0}}{\partial\mathcal{G}_{ba}^{-1}}\Xi+Z_{0}\frac{\partial\Xi}{\partial\mathcal{G}_{ba}^{-1}}\right]==\frac{1}{Z}\left[\mathcal{G}_{ab}Z_{0}\Xi-Z_{0}\Xi\mathcal{G}_{cb}\frac{1}{\Xi}\frac{\partial\Xi}{\partial\mathcal{G}_{cd}}\mathcal{G}_{ad}\right]\label{eq:G_interm}
\end{align}

Hence:
\begin{equation}
G_{ab}=\mathcal{G}_{ab}+\mathcal{G}_{ad}M_{dc}\mathcal{G}_{cb}\label{eq:G_vs_M}
\end{equation}

where we have defined:
\begin{equation}
M_{ab}\equiv-\frac{1}{\Xi}\frac{\partial\Xi}{\partial\mathcal{G}_{ba}}\label{eq:M_def}
\end{equation}

In the code, $M_{ab}$ is the observable which is measured.

We use Eq (\ref{eq:Xi_def}) to compute it as:

\[
M_{ab}=-\frac{1}{\Xi}\sum_{k\geq0,\mathbf{a},\mathbf{b}}A_{k}[{\cal U},\mathbf{a},\mathbf{b}]\frac{\partial\det\hat{\mathcal{G}}_{k}}{\partial\mathcal{G}_{ba}}
\]

Using the chain rule $\frac{\partial\det\hat{\mathcal{G}}_{k}}{\partial\mathcal{G}_{ba}}=\sum_{ij}\frac{\partial\det\hat{\mathcal{G}}_{k}}{\partial\mathcal{\hat{G}}_{ij}}\frac{\partial\hat{\mathcal{G}}_{ij}}{\partial\mathcal{G}_{ba}}$,
we obtain:

\begin{align}
M_{ab} & =-\frac{1}{\Xi}\sum_{k\geq0,\mathbf{a},\mathbf{b}}A_{k}[{\cal U},\mathbf{a},\mathbf{b}]\partial\det\hat{\mathcal{G}}_{k}\sum_{ij}\hat{\mathcal{G}}_{ji}^{-1}\delta_{a_{i}b}\delta_{b_{j}a}\nonumber \\
 & =-\frac{1}{\Xi}\sum_{\mathcal{C}}w_{\mathcal{C}}\left\{ \sum_{ij}\hat{\mathcal{G}}_{ji}^{-1}\delta_{a_{i}b}\delta_{b_{j}a}\right\} \nonumber \\
 & =\Big\langle-\sum_{ij}\hat{\mathcal{G}}_{ji}^{-1}\delta_{a_{i}b}\delta_{b_{j}a}\Big\rangle_{\mathrm{MC}}\label{eq:M_MC_final}
\end{align}

The MC average corresponds to a stochastic sum over configurations,
such that the probability of visiting a given configuration is $w({\cal C})$.

\paragraph{Implementation details}

Furthermore, in practice, on measures the binned version of $M$,
in points $\tau_{l}$ separated by $\Delta\tau$ 
\begin{eqnarray*}
\tilde{M}_{uv}(\tau_{l}) & = & -\frac{1}{\beta\Delta\tau\langle\mathrm{sgn}({\cal C})\rangle_{\mathrm{MC}}}\left\langle \sum_{ij}\delta_{ub_{i}}\delta_{va_{j}}\left(\tilde{\delta}(\tau_{l},\tau'_{i}-\tau_{j})-\tilde{\delta}(\tau_{l},\beta-\tau_{j}+\tau'_{i})\right)[\hat{\mathcal{G}}_{k}^{-1}]_{ij}\mathrm{sgn}({\cal C})\right\rangle _{\mathrm{MC}}
\end{eqnarray*}
where $\tilde{\delta}$ is the approximate Kronecker delta checking
whether the time difference falls into the given bin, and one restricts
oneself to only positive times 
\[
0\leq\tau_{l}\leq\beta
\]
When there are non-density-density time dependent interaction, times
$\tau_{i}$ and $\tau'_{i}$ may not be the same. Here $i,j$ are
rows/columns of the $\hat{\mathcal{G}}^{-1}$ matrix, corresponding
to pairs of inserted operators $c$ and $c^{\dagger}$ at corresponding
$a_{i}$ and $b_{j}$ with $\tau_{i}\equiv\tau_{a_{i}}$ and $\tau'_{j}\equiv\tau_{b_{j}}$.
Note the extra minus sign in front of the second $\tilde{\delta}$
- because the time-diff is negative, we're measuring $-M$.

The prefactor $1/\Delta\tau$ comes from the fact that for any binned
quantity $\tilde{A}(\tau_{l})=\frac{1}{\Delta\tau}\int_{\tau_{l}-\Delta\tau/2}^{\tau_{l}+\Delta\tau/2}\mathrm{d}\tau A(\tau)$
The extra $1/\beta$ comes because we are visiting the same $\tau_{l}$
many times $\int\mathrm{d}\tau A(\tau)=\frac{1}{\beta}\int\mathrm{d}\tau\mathrm{d}\tau'A(\tau-\tau')$.

\paragraph{Frequency measure}

We can also measure directly in frequency, and directly $G$ 
\begin{eqnarray*}
G_{uv}(i\omega_{n}) & = & {\cal G}_{uv}(i\omega_{n})-\frac{1}{\beta\langle\mathrm{sgn}({\cal C})\rangle_{\mathrm{MC}}}\left\langle \sum_{ab}\sum_{ij}\delta_{bb_{i}}\delta_{aa_{j}}e^{i\omega_{n}(\tau'_{i}-\tau_{j})}{\cal G}_{ub}(i\omega_{n})[\hat{\mathcal{G}}_{k}^{-1}]_{ij}{\cal G}_{av}(i\omega_{n})\mathrm{sgn}({\cal C})\right\rangle _{\mathrm{MC}}
\end{eqnarray*}
Here, there's no binning, so there's no $1/\Delta\tau$, but there
is $1/\beta$ for the same reason.

Note however, that when we are measuring 
\[
F_{uv}=M_{ua}{\cal G}_{av}
\]
there is an additional integral over time which we do not do explicitly
calculate, but sample also within the monte carlo 
\begin{eqnarray}
F(\tau_{l}) & = & \int\mathrm{d}\tau M(\tau_{l}-\tau){\cal G}(\tau)\\
 & = & \Delta\tau\sum_{j}\tilde{M}(\tau_{l}-\tau_{j}){\cal G}(\tau_{j})\\
 & = & \frac{\beta}{N_{\tau}}\sum_{j}\tilde{M}(\tau_{l}-\tau_{j}){\cal G}(\tau_{j})\\
 & = & \beta\langle\tilde{M}(\tau_{l}-\tau_{j}){\cal G}(\tau_{j})\rangle
\end{eqnarray}
so the $1/\beta$ prefactor is cancelled. 
\begin{eqnarray}
F_{uv}(\tau_{l}) & = & -\frac{1}{\Delta\tau}\left\langle \sum_{ij}\delta_{ub_{j}}\tilde{\delta}(\tau_{l},\tau'_{j})[\hat{\mathcal{G}}_{k}^{-1}]_{ji}{\cal G}_{a_{i}v}(\tau_{i})\mathrm{sgn}({\cal C})\right\rangle _{\mathrm{MC}}
\end{eqnarray}


\subsubsection{Four-point correlator $\chi^{4}$\label{subsec:Measuring-the-4-point}}

Starting from Eqs (\ref{eq:chi4_as_derivative_of_Z}), (\ref{eq:G_as_derivative_of_Z})
and (\ref{eq:df_dG_vs_df_dinvG}), we first note:

\begin{equation}
\chi_{abef}^{4}=\frac{1}{Z}\frac{\partial(ZG_{ba})}{\partial\mathcal{G}_{ef}^{-1}}=-\frac{1}{Z}\mathcal{G}_{ue}\frac{\partial(ZG_{ba})}{\partial\mathcal{G}_{uv}}\mathcal{G}_{fv}\label{eq:chi4_interm}
\end{equation}

Let us now evaluate $\frac{\partial(ZG_{ba})}{\partial\mathcal{G}_{uv}}$
using (\ref{eq:G_interm}). We have (using (\ref{eq:dZ0_dG})):

\begin{align}
\frac{\partial(ZG_{ba})}{\partial\mathcal{G}_{uv}} & =\frac{\partial}{\partial\mathcal{G}_{uv}}\left[\mathcal{G}_{ba}Z_{0}\Xi-Z_{0}\mathcal{G}_{ca}\frac{\partial\Xi}{\partial\mathcal{G}_{cd}}\mathcal{G}_{bd}\right]\nonumber \\
 & =\delta_{bu}\delta_{va}Z_{0}\Xi+\mathcal{G}_{ba}\frac{\partial Z_{0}}{\partial\mathcal{G}_{uv}}\Xi+\mathcal{G}_{ba}Z_{0}\frac{\partial\Xi}{\partial\mathcal{G}_{uv}}\nonumber \\
 & \;\;-\frac{\partial Z_{0}}{\partial\mathcal{G}_{uv}}\mathcal{G}_{ca}\frac{\partial\Xi}{\partial\mathcal{G}_{cd}}\mathcal{G}_{bd}-Z_{0}\delta_{uc}\delta_{av}\frac{\partial\Xi}{\partial\mathcal{G}_{cd}}\mathcal{G}_{bd}-Z_{0}\mathcal{G}_{ca}\frac{\partial\Xi}{\partial\mathcal{G}_{uv}\partial\mathcal{G}_{cd}}\mathcal{G}_{bd}-Z_{0}\mathcal{G}_{ca}\frac{\partial\Xi}{\partial\mathcal{G}_{cd}}\delta_{bu}\delta_{dv}\nonumber \\
 & =\delta_{bu}\delta_{va}Z_{0}\Xi-\mathcal{G}_{ba}\mathcal{G}_{vu}^{-1}Z_{0}\Xi-\mathcal{G}_{ba}Z_{0}\Xi M_{vu}\nonumber \\
 & \;\;-\mathcal{G}_{vu}^{-1}Z_{0}\mathcal{G}_{ca}\Xi M_{dc}\mathcal{G}_{bd}+\Xi Z_{0}\delta_{uc}\delta_{av}M_{dc}\mathcal{G}_{bd}-Z_{0}\mathcal{G}_{ca}\Xi M_{vudc}^{4}\mathcal{G}_{bd}+Z_{0}\mathcal{G}_{ca}\Xi M_{dc}\delta_{bu}\delta_{dv}\nonumber \\
 & =Z\Big\{\delta_{bu}\delta_{va}-\mathcal{G}_{ba}\mathcal{G}_{vu}^{-1}-\mathcal{G}_{ba}M_{vu}-\mathcal{G}_{vu}^{-1}\mathcal{G}_{ca}M_{dc}\mathcal{G}_{bd}\nonumber \\
 & \;\;\;+\delta_{av}M_{du}\mathcal{G}_{bd}-\mathcal{G}_{ca}M_{uvcd}^{4}\mathcal{G}_{bd}+\mathcal{G}_{ca}M_{vc}\delta_{bu}\Big\}\label{eq:chi4_interm_2}
\end{align}

where we have defined:

\begin{equation}
M_{u\bar{v}c\bar{d}}^{4}\equiv\frac{1}{\Xi}\frac{\partial\Xi}{\partial\mathcal{G}_{u\bar{v}}\partial\mathcal{G}_{c\bar{d}}}\label{eq:M4_def}
\end{equation}

Plugging (\ref{eq:chi4_interm_2}) into (\ref{eq:chi4_interm}), we
obtain
\begin{align*}
\chi_{abef}^{4} & =-\mathcal{G}_{be}\mathcal{G}_{fa}+\mathcal{G}_{fe}\mathcal{G}_{ba}+\mathcal{G}_{ba}\mathcal{G}_{fv}M_{vu}\mathcal{G}_{ue}+\mathcal{G}_{bd}M_{dc}\mathcal{G}_{ca}\mathcal{G}_{fe}\\
 & \;\;-\mathcal{G}_{bd}M_{du}\mathcal{G}_{ue}\mathcal{G}_{fa}+\mathcal{G}_{fv}\mathcal{G}_{ue}M_{uvcd}^{4}\mathcal{G}_{bd}\mathcal{G}_{ca}-\mathcal{G}_{be}\mathcal{G}_{fv}M_{vc}\mathcal{G}_{ca}
\end{align*}

i.e, after relabeling and reordering:

\begin{align}
\chi_{\bar{a}b\bar{c}d}^{4} & =\mathcal{G}_{d\bar{v}}\mathcal{G}_{u\bar{c}}M_{u\bar{v}x\bar{y}}^{4}\mathcal{G}_{b\bar{y}}\mathcal{G}_{x\bar{a}}\nonumber \\
 & \;\;+\mathcal{G}_{b\bar{a}}\mathcal{G}_{d\bar{c}}+[\mathcal{G}M\mathcal{G}]_{b\bar{a}}\mathcal{G}_{d\bar{c}}+\mathcal{G}_{b\bar{a}}[\mathcal{G}M\mathcal{G}]_{d\bar{c}}\nonumber \\
 & \;\;-\mathcal{G}_{b\bar{c}}\mathcal{G}_{d\bar{a}}-[\mathcal{G}M\mathcal{G}]_{b\bar{c}}\mathcal{G}_{d\bar{a}}-\mathcal{G}_{b\bar{c}}[\mathcal{G}M\mathcal{G}]_{d\bar{a}}\label{eq:chi4_final}
\end{align}

Thus, in order to compute $\chi^{4}$, one needs to compute $M$ and
$M^{4}$.

\paragraph{Formulation in frequency}

Let us rewrite the terms of Eq. (\ref{eq:chi4_final}) in frequency
space:

\begin{align*}
\chi_{abcd}^{4,0}(i\omega,i\omega',i\Omega) & \equiv\iiiint_{1234}e^{-i\omega\tau_{1}}e^{(i\omega+i\Omega)\tau_{2}}e^{-(i\omega'+i\Omega)\tau_{3}}e^{i\omega'\tau_{4}}\\
 & \;\;\;\iiiint_{5678}{\cal G}_{uc}(\tau_{5}-\tau_{3}){\cal G}_{dv}(\tau_{4}-\tau_{6})M_{uvxy}^{4}(\tau_{5},\tau_{6},\tau_{7},\tau_{8}){\cal G}_{xa}(\tau_{7}-\tau_{1}){\cal G}_{by}(\tau_{2}-\tau_{8})\\
 & =\sum_{\omega_{1}\omega_{2}\omega_{3}\omega_{3}'\Omega_{3}\omega_{4}\omega_{5}}\iiiint_{1\dots8}\left\{ e^{-i\omega\tau_{1}}e^{(i\omega+i\Omega)\tau_{2}}e^{-(i\omega'+i\Omega)\tau_{3}}e^{i\omega'\tau_{4}}\right\} e^{-i\omega_{1}(\tau_{5}-\tau_{3})}e^{-i\omega_{2}(\tau_{4}-\tau_{6})}\\
 & \;\;\;e^{-i\omega_{4}(\tau_{7}-\tau_{1})}e^{-i\omega_{5}(\tau_{2}-\tau_{8})}e^{i\omega_{3}\tau_{5}-(i\omega_{3}+i\Omega_{3})\tau_{6}+(i\omega_{3}'+i\Omega_{3})\tau_{7}-i\omega_{3}'\tau_{8}}\\
 & \;\;\;{\cal G}_{uc}(i\omega_{1}){\cal G}_{dv}(i\omega_{2})M_{uvxy}^{4}(i\omega_{3},i\omega_{3}',i\Omega{}_{3}){\cal G}_{xa}(i\omega_{4}){\cal G}_{by}(i\omega_{5})\\
 & =\sum_{\omega_{1}\omega_{2}\omega_{3}\omega_{3}'\Omega_{3}\omega_{4}\omega_{5}}\iiiint_{1\dots8}e^{i\tau_{1}(-\omega+\omega_{4})+i\tau_{2}(\omega+\Omega-\omega_{5})+i\tau_{3}(-\omega'-\Omega'+\omega_{1})+i\tau_{4}(\omega'-\omega_{2})}\\
 & \;\;\;e^{i\tau_{5}(-\omega_{1}+\omega_{3})+i\tau_{6}(\omega_{2}-\omega_{3}-\Omega_{3})+i\tau_{7}(-\omega_{4}+\omega_{3}'+\Omega_{3})+i\tau_{8}(\omega_{5}-\omega_{3}')}\\
 & \;\;\;{\cal G}_{uc}(i\omega_{1}){\cal G}_{dv}(i\omega_{2})M_{uvxy}^{4}(i\omega_{3},i\omega_{3}',i\Omega{}_{3}){\cal G}_{xa}(i\omega_{4}){\cal G}_{by}(i\omega_{5})\\
 & ={\cal G}_{uc}(\omega'+\Omega'){\cal G}_{dv}(\omega')M_{uvxy}^{4}(\omega'+\Omega',\omega+\Omega,-\Omega){\cal G}_{xa}(\omega){\cal G}_{by}(\omega+\Omega)
\end{align*}

\begin{align*}
\chi_{abcd}^{4,1}(i\omega,i\omega',i\Omega) & \equiv\iiiint_{1234}e^{-i\omega\tau_{1}}e^{(i\omega+i\Omega)\tau_{2}}e^{-(i\omega'+i\Omega)\tau_{3}}e^{i\omega'\tau_{4}}\mathcal{G}_{ba}(\tau_{2}-\tau_{1})\mathcal{G}_{dc}(\tau_{4}-\tau_{3})\\
 & =\iiiint_{1234}e^{-i\omega\tau_{1}}e^{(i\omega+i\Omega)\tau_{2}}e^{-(i\omega'+i\Omega)\tau_{3}}e^{i\omega'\tau_{4}}\iiiint_{5678}\\
 & \;\;\sum_{\omega_{1},\omega_{2}}e^{-i\omega_{1}(\tau_{2}-\tau_{1})-i\omega_{2}(\tau_{4}-\tau_{3})}\mathcal{G}_{ba}(i\omega_{1})\mathcal{G}_{dc}(i\omega_{2})\\
 & =\sum_{\omega_{1},\omega_{2}}\iiiint_{1234}e^{i\tau_{1}(-\omega+\omega_{1})+i\tau_{2}(\omega+\Omega-\omega_{1})+i\tau_{3}(-\omega'-\Omega+\omega_{2})+i\tau_{4}(\omega'-\omega_{2})}\\
 & \;\;\mathcal{G}_{ba}(\omega)\mathcal{G}_{dc}(\omega')\beta\delta_{\Omega}
\end{align*}

\begin{align*}
\chi_{abcd}^{4,2}i\omega,i\omega',i\Omega) & \equiv\iiiint_{1234}e^{-i\omega\tau_{1}}e^{(i\omega+i\Omega)\tau_{2}}e^{-(i\omega'+i\Omega)\tau_{3}}e^{i\omega'\tau_{4}}\mathcal{G}_{bc}(\tau_{2}-\tau_{3})\mathcal{G}_{da}(\tau_{4}-\tau_{1})\\
 & =\sum_{\omega_{1}\omega_{2}}\iiiint_{1234}e^{-i\omega\tau_{1}}e^{(i\omega+i\Omega)\tau_{2}}e^{-(i\omega'+i\Omega)\tau_{3}}e^{i\omega'\tau_{4}}e^{-i\omega_{1}(\tau_{2}-\tau_{3})}e^{-i\omega_{2}(\tau_{4}-\tau_{1})}\mathcal{G}_{bc}(\omega_{1})\mathcal{G}_{da}(\omega_{2})\\
 & =\sum_{\omega_{1}\omega_{2}}\iiiint_{1234}e^{i\tau_{1}(-\omega+\omega_{2})}e^{i\tau_{2}(\omega+\Omega-\omega_{1})}e^{\tau_{3}(-\omega'-\Omega+\omega_{1})}e^{i\tau_{4}(\omega'-\omega_{2})}\mathcal{G}_{bc}(\omega_{1})\mathcal{G}_{da}(\omega_{2})\\
 & =\mathcal{G}_{bc}(\omega+\Omega)\mathcal{G}_{da}(\omega)\beta\delta_{\omega,\omega'}
\end{align*}

Thus, we obtain the final expression:

\begin{align}
\chi_{abcd}^{4,0}(i\omega,i\omega',i\Omega) & ={\cal G}_{uc}(\omega'+\Omega){\cal G}_{dv}(\omega')M_{uvxy}^{4}(\omega'+\Omega,\omega+\Omega,-\Omega){\cal G}_{xa}(\omega){\cal G}_{by}(\omega+\Omega)\label{eq:chi4_final_freq}\\
 & \;\;+\mathcal{G}_{ba}(\omega)\mathcal{G}_{dc}(\omega')\beta\delta_{\Omega}+[\mathcal{G}M\mathcal{G}]_{ba}(\omega)\mathcal{G}_{dc}(\omega')\beta\delta_{\Omega}+\mathcal{G}_{ba}(\omega)[\mathcal{G}M\mathcal{G}]_{dc}(\omega')\beta\delta_{\Omega}\nonumber \\
 & \;\;-\mathcal{G}_{bc}(\omega+\Omega)\mathcal{G}_{da}(\omega)\beta\delta_{\omega,\omega'}-[\mathcal{G}M\mathcal{G}]_{bc}(\omega+\Omega)\mathcal{G}_{da}(\omega)\beta\delta_{\omega,\omega'}-\mathcal{G}_{bc}(\omega+\Omega)[\mathcal{G}M\mathcal{G}]_{da}(\omega)\beta\delta_{\omega,\omega'}\nonumber 
\end{align}

Or, in terms of $\tilde{\chi}^{4}$ (see Eq.(\ref{eq:chi4_tilde_def})
for a definition and (\ref{eq:chi4tilde_chi4_freq}) for the relation
to $\chi^{4}$):

\begin{align}
\tilde{\chi}_{abcd}^{4,0}(i\omega,i\omega',i\Omega) & ={\cal G}_{ud}(\omega'){\cal G}_{cv}(\omega'+\Omega)M_{uvxy}^{4}(\omega',\omega,\Omega){\cal G}_{xb}(\omega+\Omega){\cal G}_{ay}(\omega)\label{eq:chi4_final_freq-1}\\
 & \;\;+\left\{ \mathcal{G}_{ab}(\omega)\mathcal{G}_{cd}(\omega')+[\mathcal{G}M\mathcal{G}]_{ab}(\omega)\mathcal{G}_{cd}(\omega')+\mathcal{G}_{ab}(\omega)[\mathcal{G}M\mathcal{G}]_{cd}(\omega')\right\} \beta\delta_{\Omega}\nonumber \\
 & \;\;-\left\{ \mathcal{G}_{ad}(\omega)\mathcal{G}_{cb}(\omega+\Omega)+[\mathcal{G}M\mathcal{G}]_{ad}(\omega)\mathcal{G}_{cb}(\omega+\Omega)+\mathcal{G}_{ad}(\omega)[\mathcal{G}M\mathcal{G}]_{cb}(\omega+\Omega)\right\} \beta\delta_{\omega,\omega'}\nonumber 
\end{align}

\begin{lyxgreyedout}
Note that this is (a priori) different from the previous version of
the notes ($\omega'$ and $\omega$ swapped in $M^{4}$...))
\begin{eqnarray}
\chi_{\sigma\sigma'}^{4}(i\omega,i\omega',i\nu) & = & \Big({\cal G}_{\sigma}(i\omega){\cal G}_{\sigma'}(i\omega')+[{\cal G}M{\cal G}]_{\sigma}(i\omega){\cal G}_{\sigma'}(i\omega')+{\cal G}_{\sigma}(i\omega)[{\cal G}M{\cal G}]_{\sigma'}(i\omega')\Big)\delta_{i\nu,0}\\
 &  & -\Big({\cal G}_{\sigma}(i\omega){\cal G}_{\sigma}(i\omega+i\nu)+[{\cal G}M{\cal G}]_{\sigma}(i\omega){\cal G}_{\sigma}(i\omega+i\nu)+{\cal G}_{\sigma}(i\omega)[{\cal G}M{\cal G}]_{\sigma}(i\omega+i\nu)\Big)\delta_{\sigma,\sigma'}\delta_{i\omega,i\omega'}\nonumber \\
 &  & +{\cal G}_{\sigma}(i\omega){\cal G}_{\sigma}(i\omega+i\nu)M_{\sigma\sigma'}^{4}(i\omega,i\omega',i\nu){\cal G}_{\sigma'}(i\omega'){\cal G}_{\sigma'}(i\omega'+i\nu)\label{chi4toM4}
\end{eqnarray}
\end{lyxgreyedout}

At the moment, what is implemented in the code is $\tilde{\chi}_{a\bar{b}c\bar{d}}^{4}(i\omega,i\omega',i\Omega)$.

\paragraph{Connected four-point function}

The connected correlation function (see Eq(\ref{eq:chi4tilde_conn_def}))
is:
\begin{align*}
\tilde{\chi}_{a\bar{b}c\bar{d}}^{4,\mathrm{conn}} & =\tilde{\chi}_{abcd}^{4}-\left[\mathcal{G}+\mathcal{G}M\mathcal{G}\right]_{ab}\left[\mathcal{G}+\mathcal{G}M\mathcal{G}\right]_{cd}+\left[\mathcal{G}+\mathcal{G}M\mathcal{G}\right]_{ad}\left[\mathcal{G}+\mathcal{G}M\mathcal{G}\right]_{cb}\\
 & =\tilde{\chi}_{abcd}^{4}\\
 & \;\;-\left\{ \mathcal{G}_{ab}\mathcal{G}_{cd}+\mathcal{G}_{ab}\left[\mathcal{G}M\mathcal{G}\right]_{cd}+\left[\mathcal{G}M\mathcal{G}\right]_{ab}\mathcal{G}_{cd}+\left[\mathcal{G}M\mathcal{G}\right]_{ab}\left[\mathcal{G}M\mathcal{G}\right]_{cd}\right\} \\
 & \;\;+\left\{ \mathcal{G}_{ad}\mathcal{G}_{cb}+\mathcal{G}_{ad}\left[\mathcal{G}M\mathcal{G}\right]_{cb}+\left[\mathcal{G}M\mathcal{G}\right]_{ad}\mathcal{G}_{cb}+\left[\mathcal{G}M\mathcal{G}\right]_{ad}\left[\mathcal{G}M\mathcal{G}\right]_{cb}\right\} 
\end{align*}

By inspection of Eq.(\ref{eq:chi4_final}), we see that

\begin{align}
\tilde{\chi}_{a\bar{b}c\bar{d}}^{4,\mathrm{conn}} & =\mathcal{G}_{c\bar{v}}\mathcal{G}_{u\bar{d}}M_{u\bar{v}x\bar{y}}^{4}\mathcal{G}_{a\bar{y}}\mathcal{G}_{x\bar{b}}\nonumber \\
 & \;\;-\left[\mathcal{G}M\mathcal{G}\right]_{a\bar{b}}\left[\mathcal{G}M\mathcal{G}\right]_{c\bar{d}}+\left[\mathcal{G}M\mathcal{G}\right]_{a\bar{d}}\left[\mathcal{G}M\mathcal{G}\right]_{c\bar{b}}\label{eq:chi4tilde_conn_final}
\end{align}

Or in other words:

\[
\tilde{\chi}_{a\bar{b}c\bar{d}}^{4,\mathrm{conn}}=\mathcal{G}_{c\bar{v}}\mathcal{G}_{u\bar{d}}M_{u\bar{v}x\bar{y}}^{4,\mathrm{conn}}\mathcal{G}_{a\bar{y}}\mathcal{G}_{x\bar{b}}
\]

where we have defined:

\[
M_{u\bar{v}x\bar{y}}^{4,\mathrm{conn}}\equiv M_{u\bar{v}x\bar{y}}^{4}-M_{\bar{v}u}M_{\bar{y}x}+M_{\bar{v}x}M_{\bar{y}u}
\]

In frequency space:

\begin{align}
\tilde{\chi}_{a\bar{b}c\bar{d}}^{4,\mathrm{conn}}(i\omega,i\omega',i\Omega) & ={\cal G}_{ud}(\omega'){\cal G}_{cv}(\omega'+\Omega)M_{uvxy}^{4}(\omega',\omega,\Omega){\cal G}_{xb}(\omega+\Omega){\cal G}_{ay}(\omega)\nonumber \\
 & \;\;-[\mathcal{G}M\mathcal{G}]_{ab}(\omega)[\mathcal{G}M\mathcal{G}]_{cd}(\omega')\beta\delta_{\Omega}+[\mathcal{G}M\mathcal{G}]_{ad}(\omega)[\mathcal{G}M\mathcal{G}]_{cb}(\omega+\Omega)\beta\delta_{\omega,\omega'}\label{eq:chi4tilde_conn_final_freq}
\end{align}

or

\[
\tilde{\chi}_{a\bar{b}c\bar{d}}^{4,\mathrm{conn}}(i\omega,i\omega',i\Omega)={\cal G}_{u\bar{d}}(\omega'){\cal G}_{c\bar{v}}(\omega'+\Omega)M_{uvxy}^{4,\mathrm{conn}}(\omega',\omega,\Omega){\cal G}_{x\bar{b}}(\omega+\Omega){\cal G}_{a\bar{y}}(\omega)
\]
\label{eq:chi4tilde_conn_final_freq_v2}

with
\[
M_{u\bar{v}x\bar{y}}^{4,\mathrm{conn}}(i\omega,i\omega',i\Omega)=M_{u\bar{v}x\bar{y}}^{4}(i\omega,i\omega',i\Omega)-M_{\bar{v}u}(\omega)M_{\bar{y}x}(\omega')\beta\delta_{\Omega}+M_{\bar{v}x}(\omega+\Omega)M_{\bar{y}u}(\omega)\beta\delta_{\omega,\omega'}
\]

Eq. (\ref{eq:chi4tilde_conn_final_freq_v2}) is a bit faster in terms
of computing time than (\ref{eq:chi4tilde_conn_final_freq}). %
\begin{lyxgreyedout}
Sampling directly $M^{4,\mathrm{conn}}$ (instead of $M^{4}$) probably
leads to better statistics.%
\end{lyxgreyedout}


\paragraph{Fully reducible vertex\label{par:Fully-reducible-vertex}}

Using (\ref{eq:F_def}), we get

\[
F_{\bar{a}b\bar{c}d}=\left\{ G_{\bar{a}a}^{-1}{\cal G}_{a\bar{y}}\right\} \left\{ {\cal G}_{x\bar{b}}G_{\bar{b}b}^{-1}\right\} M_{uvxy}^{4,\mathrm{conn}}\left\{ G_{\bar{c}c}^{-1}{\cal G}_{c\bar{v}}\right\} \left\{ {\cal G}_{u\bar{d}}G_{\bar{d}d}^{-1}\right\} 
\]

i.e, recalling (\ref{eq:G_vs_M}),

\begin{equation}
F_{\bar{a}b\bar{c}d}=\left\{ 1+M\mathcal{G}\right\} _{\bar{a}\bar{y}}^{-1}\left\{ 1+M\mathcal{G}\right\} _{xb}^{-1}M_{u\bar{v}x\bar{y}}^{4,\mathrm{conn}}\left\{ 1+M\mathcal{G}\right\} _{\bar{c}\bar{v}}^{-1}\left\{ 1+M\mathcal{G}\right\} _{ud}^{-1}\label{eq:F_final}
\end{equation}

From this expression, one can see that $F$ and $M^{4,\mathrm{conn}}$
have the same asymptotics.

\begin{lyxgreyedout}
Could one sample $F$ directly? Is it numerically better?%
\end{lyxgreyedout}


\paragraph{Implementation: measurement of $M^{4}$}

One measures: 
\begin{eqnarray}
\frac{\delta^{2}\xi}{\delta{\cal G}_{a_{j}b_{i}}\delta{\cal G}_{a_{l}b_{k}}} & \sim & \frac{\delta^{2}\det\hat{\mathcal{G}}}{\delta\hat{\mathcal{G}}_{ji}\delta\hat{\mathcal{G}}_{lk}}\nonumber \\
 & = & \left([\hat{\mathcal{G}}^{-1}]_{ij}[\hat{\mathcal{G}}^{-1}]_{kl}-[\hat{\mathcal{G}}^{-1}]_{il}[\hat{\mathcal{G}}^{-1}]_{kj}\right)\det\hat{\mathcal{G}}
\end{eqnarray}
so 
\begin{eqnarray}
M_{abcd}^{4}\sim\frac{1}{\langle\mathrm{sgn}(\mathcal{C})\rangle_{\mathrm{MC}}}\Big\langle\sum_{ijkl}\Big([\hat{\mathcal{G}}^{-1}]_{ij}[\hat{\mathcal{G}}^{-1}]_{kl}-[\hat{\mathcal{G}}^{-1}]_{il}[\hat{\mathcal{G}}^{-1}]_{kj}\Big)\delta_{aa_{j}}\delta_{bb_{i}}\delta_{ca_{l}}\delta_{db_{k}}\mathrm{sgn}(\mathcal{C})\Big\rangle_{\mathrm{MC}}
\end{eqnarray}

In practice, to avoid doing a quadruple loop over operators ($ijkl$),
for a given configuration we can define a ``time-translationaly broken
scattering matrix'' 
\begin{equation}
M_{uv}(\tau,\tau')\equiv\sum_{ij}[\hat{\mathcal{G}}^{-1}]_{ji}\delta_{a_{i}u}\delta_{b_{j}v}\delta_{\tau_{i}\tau}\delta_{\tau'_{j}\tau'}
\end{equation}
and then 
\begin{equation}
M_{uvxw}^{4}(\tau_{1},\tau_{2},\tau_{3})\sim\frac{1}{\langle\mathrm{sgn}(\mathcal{C})\rangle_{\mathrm{MC}}}\Big\langle\Big(M_{uv}(\tau_{1},\tau_{2})M_{xw}(\tau_{3},0)-M_{uw}(\tau_{1},0)M_{xv}(\tau_{3},\tau_{2})\Big)\mathrm{sgn}(\mathcal{C})\Big\rangle_{\mathrm{MC}}
\end{equation}
In frequency space: 
\begin{equation}
M_{uv}(i\omega,i\omega')=\sum_{ij}[\hat{\mathcal{G}}^{-1}]_{ji}\delta_{a_{i}u}\delta_{b_{j}v}e^{i\omega\tau-i\omega'\tau'_{j}}
\end{equation}
where the minus sign in fron of $\tau'$ term in the exponent is due
to definition of Fourier transform of $FT(c^{\dagger}(\tau))=FT^{*}(c(\tau))$
\begin{equation}
M_{uvxw}^{4}(i\omega,i\omega',i\nu)\sim\Big\langle\Big(M_{uv}(i\omega,i\omega+i\nu)M_{xw}(i\omega'+i\nu,i\omega')-M_{uw}(i\omega,i\omega')M_{xv}(i\omega'+i\nu,i\omega'+i\nu)\Big)\mathrm{sgn}(\mathcal{C})\Big\rangle_{\mathrm{MC}}
\end{equation}


\subsubsection{Two-point correlation function $\chi^{2}$ \label{subsec:Measuring-the-two-point}}

The two-point correlation function is measured by operator insertion.

Let us perform the following interaction expansion: 
\begin{eqnarray}
\chi^{2} & = & \frac{1}{Z}\int{\cal D}[c^{\dagger},c]\,e^{c_{u}^{\dagger}[{\cal G}^{-1}]_{uv}c_{v}}\sum_{k,\mathbf{a},\mathbf{b}}(-)^{k}A_{k}[{\cal U},\mathbf{a},\mathbf{b}]n_{u}n_{v}c_{a_{1}}^{\dagger}...c_{a_{2k}}^{\dagger}c_{b_{1}}...c_{b_{2k}}\nonumber \\
 & = & \frac{Z_{0}}{Z}\sum_{k,\mathbf{a},\mathbf{b}}A_{k}[{\cal U},\mathbf{a},\mathbf{b}]\det D_{k}^{[u,v]}[{\cal G},\mathbf{a},\mathbf{b}]
\end{eqnarray}
where 
\begin{eqnarray}
D_{k}^{[u,v]}[{\cal G},\mathbf{a},\mathbf{b}]=\left(\begin{array}{ccccc}
{\cal G}_{uu} & {\cal G}_{uv} & {\cal G}_{ub_{1}} & ... & {\cal G}_{ub_{2k}}\\
{\cal G}_{vu} & {\cal G}_{vv} & {\cal G}_{vb_{1}} & ... & {\cal G}_{vb_{2k}}\\
{\cal G}_{a_{1}u} & {\cal G}_{a_{1}v} & {\cal G}_{a_{1}b_{1}} & ... & {\cal G}_{a_{1}b_{2k}}\\
... & ... & ... & ... & ...\\
{\cal G}_{a_{2k}u} & {\cal G}_{a_{2k}v} & ... & ... & {\cal G}_{a_{2k}b_{2k}}
\end{array}\right)
\end{eqnarray}
The determinant of this matrix can be expressed as 
\begin{eqnarray}
\det D_{k}^{[u,v]}=P\det D_{k}
\end{eqnarray}
where $D_{k}$ is defined in Eq.~\ref{defD}, while the determinant
ratio $P$ can be evaluated by the Woodbury identity and is what is
returned by try-insert-two. The effort is $O(k^{2})$ + the determinant
of $2\times2$ matrix which is negligible. In The case when $u$ and
$v$ are in different blocks, then, only one insertion is necessary
and we have for example 
\begin{equation}
\det D_{k}^{[u,v]}=\det D_{\sigma_{u}k}^{[u]}\det D_{\sigma_{v}k}^{[v]}=P_{\sigma_{v}}P_{\sigma_{u}}\det D_{\sigma_{v}k}\det D_{\sigma_{u}k}
\end{equation}

Finally, we use $\det D$ as the weight and the measurement is nothing
but 
\begin{equation}
\chi^{2}=\frac{1}{\langle\mathrm{sgn}{\cal C}\rangle_{\mathrm{MC}}}\left\langle P\right\rangle _{\mathrm{MC}}
\end{equation}

The same can be done for the 3-point correlation function $\chi_{uvw}^{3}$
defined later (Eq (\ref{eq:chi3_def-1})) and for the 4-point correlation
function $\chi_{uvwx}^{4}$ (Eq. (\ref{eq:chi4_def})). However, the
computational effort for a single measurement here is $O(k^{2})$
while by cutting lines is $O(1)$, because the determinant ratio is
only one element of the inverse matrix. However, the quality of measurement
is better controlled by doing the insertion. It becomes especially
important when we only have non-density density interactions – in
that case measurement of correlators including density operators is
not possible by cutting lines in $\Xi$.

\subsubsection{Three-point correlation function $\chi^{3}$}

\paragraph{Computation from $\chi^{4}$}

See Eq. (\ref{eq:chi3_from_chi4}).

Note: In practice we don't put $1/\beta$ in front of the summation
over Matsubara frequencies because we don't multiply the disconnected
terms in Eq.\ref{chi4toM4}, but then we must divide the connected
term by $\beta$ for everything to be consistent. 

The Fourier conventions are given in Appendix \ref{sec:Fourier-conventions}.

\paragraph{Measurement by operator insertion}

See section \ref{subsec:Measuring-the-two-point}. To be continued.

\pagebreak{}

\section{Implementation notes}

\subsection{General structure of the code}

To be continued

\subsubsection{Structure of the blocks}

\subsubsection{Structure of the interaction vertex}

\subsection{Monte-Carlo updates (moves)}

We have a list of vertices with all indices specified in advance,
including the auxiliary spin $s$. So when a vertex is chosen for
insertion one only needs to pick a vertex from a list and generate
random time(s).

\subsubsection{Move insert vertex }
\begin{itemize}
\item choose the vertex from the list 
\item if static, chose one time $\tau\in[0,\beta]$, otherwise two times
$\tau,\tau'$ 
\item add to the list of vertices in configuration 
\end{itemize}
Depending on whether the vertex is static or dynamic, the probability
to choose the given vertex and the corresponding times is 
\begin{eqnarray*}
P^{i\rightarrow f} & = & \frac{1}{N_{ver}\beta^{(2)}}\\
\end{eqnarray*}
Note that the $N_{s}$ is already included in $N_{ver}$, because
density-density vertices appear in $N_{s}$ copies with each possible
$s$. To go back, one just needs to choose this one out of all the
vertices present in the configuration, which there will be $k+1$
when we insert the chosen vertex, at the present (total) perturbation
order $k$ 
\begin{eqnarray*}
P^{f\rightarrow i} & = & \frac{1}{k+1}
\end{eqnarray*}
The weight ratio is 
\[
w=-\frac{N_{ver}\beta^{(2)}{\cal U}_{uv}}{k+1}\times\frac{\det\hat{\mathcal{G}}^{[u,v]}}{\det\hat{\mathcal{G}}}
\]
where ${\cal U}_{uv}$ is the interaction amplitude. For example if
the vertex is dynamic density-density, it is ${\cal U}_{uv}={\cal D}_{uv}/4$.
{\color{red} Why am I not missing here $1/N_{s}$ in the amplitude?
} Note also the minus sign in front which comes from $(-)^{k}$.
However, an extra minus sigh should be put in ${\cal J}^{\perp}$
because the sign does NOT depend on the perturbation order in ${\cal J}^{\perp}$.
Note that in the code one does not see $k+1$ but only $k$ because
at that point, the perturbation order is already increased (the vertex
is already added to the list of vertices in the configuration, and
if the move is not acepted will be removed from it). If accepted: 
\begin{itemize}
\item complete insertions in determinants 
\item together with the amliptude ${\cal U}_{uv}$ to be used when vertex
is evenutally removed. 
\end{itemize}
If rejected: 
\begin{itemize}
\item remove the vertex from the list of vertices present in the configuration 
\end{itemize}

\subsubsection{Move remove vertex}
\begin{itemize}
\item choose one vertex among the vertices present in the configuration
(double removal if same block, remove in different determinants if
different blocks) 
\end{itemize}
If accepted 
\begin{itemize}
\item complete operations on determinants 
\item erase vertex from the list of vertices in the configuration 
\end{itemize}
The weight ratio is 
\[
w=-\frac{k}{N_{ver}\beta^{(2)}{\cal U}_{uv}}\times\frac{\det\hat{\mathcal{G}}^{(u,v)}}{\det D}
\]


\subsection{Monte-Carlo measures}

The measures can be found in the folder c++/measures. One measure
is performed by class. The list of all measures is:
\begin{enumerate}
\item measure\_chi3pmt ($\chi_{\pm}^{2}[\sigma,\sigma'](\tau)$)
\item measure\_ft ($F_{ab}[\sigma](\tau)$)
\item measure\_gw ($G_{ab}[\sigma](i\omega)$)
\item measure\_M4t ($M_{abcd}^{4}[\sigma,\sigma'](\tau,\tau',\tau'')$)
\item measure\_Mt ($M_{ab}[\sigma](\tau)$)
\item measure\_nnt ($\chi_{ab}^{2}[\sigma,\sigma'](\tau)$)
\item measure\_chipmt
\item measure\_g2t ($\chi_{abc}^{3}[\sigma,\sigma'](\tau,\tau')$)
\item measure\_hist: measure of the histogram of expansion oder.
\item measure\_M4w
\item measure\_nn ($\chi_{ab}^{2}[\sigma,\sigma']$)
\item measure\_sign: measure of the Monte-Carlo sign
\end{enumerate}
\pagebreak{}

\section{Future development and known issues}

\subsection{Future development}

\subsection{Known issues}

\pagebreak{}

\appendix

\section{Details of derivations\label{sec:Details}}

\subsection{Derivative of inverse matrix\label{subsec:Derivative-of-inverse}}

\subsection{Derivative of determinant \label{subsec:Derivative-of-determinant}}

This can be proven the following way. We know the grassman gaussian
integral 
\begin{equation}
\int{\cal D}[c^{\dagger},c]e^{-c_{u}^{+}A_{uv}c_{v}}=\det\mathbf{A}
\end{equation}
So we can use this 
\begin{eqnarray}
\frac{\delta}{\delta A_{ij}}\det\mathbf{A} & = & \frac{\delta}{\delta A_{ij}}\int{\cal D}[c^{\dagger},c]e^{-c_{u}^{+}A_{uv}c_{v}}\\
 & = & \int{\cal D}[c^{\dagger},c](-c_{i}^{+}c_{j})e^{-c_{u}^{+}A_{uv}c_{v}}\\
 & = & -\det\mathbf{A}\langle c_{i}^{+}c_{j}\rangle
\end{eqnarray}
where the average is taken with respect to the action 
\[
S=c_{u}^{+}A_{uv}c_{v}
\]
and is by definition the bare propagator ${\cal G}_{ji}=\langle c_{i}^{+}c_{j}\rangle$.
However, we know that ${\cal G}=-\mathbf{A}^{-1}$ so 
\begin{eqnarray}
\frac{\delta}{\delta A_{ij}}\det\mathbf{A} & = & [\mathbf{A}^{-1}]_{ji}\,\det\mathbf{A}
\end{eqnarray}


\section{Fourier conventions\label{sec:Fourier-conventions}}

We define: 
\begin{equation}
\chi_{abcd}^{4}(i\omega,i\omega',i\nu)\equiv\langle c_{a}(i\omega)c_{b}^{\dagger}(i\omega+i\nu)c_{c}(i\omega'+i\nu)c_{d}^{+}(i\omega')\rangle
\end{equation}
as the Fourier transform of 
\begin{equation}
\chi_{abcd}^{4}(\tau_{1}-\tau_{2},\tau_{3}-\tau_{4},\tau_{2}-\tau_{3})\equiv\langle Tc_{a}(\tau_{1})c_{b}^{\dagger}(\tau_{2})c_{c}(\tau_{3})c_{d}^{+}(\tau_{4})\rangle\label{eq:chi4_def-1}
\end{equation}
We can show it the following way 
\begin{equation}
\chi_{abcd}^{4}(i\omega_{1},i\omega_{2},i\omega_{3},i\omega_{4})=\int\prod_{i=1}^{4}\mathrm{d}\tau e^{i\omega_{1}\tau_{1}-i\omega_{2}\tau_{2}+i\omega_{3}\tau_{3}-i\omega_{4}\tau_{4}}\langle Tc_{a}(\tau_{1})c_{b}^{\dagger}(\tau_{2})c_{c}(\tau_{3})c_{d}^{+}(\tau_{4})\rangle
\end{equation}
Now we impose energy conservation 
\begin{equation}
i\omega_{1}+i\omega_{3}=i\omega_{2}+i\omega_{4}
\end{equation}
so we can define a bosonic frequency 
\begin{equation}
i\nu=i\omega_{2}-i\omega_{1}=i\omega_{4}-i\omega_{3}
\end{equation}
\begin{eqnarray*}
\chi_{abcd}^{4}(i\omega_{1},i\omega_{4},i\nu) & = & \frac{1}{\beta}\int\prod_{i=1}^{4}\mathrm{d}\tau e^{i\omega_{1}\tau_{1}-(i\omega_{1}+i\nu)\tau_{2}+(i\omega_{4}-i\nu)\tau_{3}-i\omega_{4}\tau_{4}}\langle Tc_{a}(\tau_{1})c_{b}^{\dagger}(\tau_{2})c_{c}(\tau_{3})c_{d}^{+}(\tau_{4})\rangle\\
 & = & \frac{1}{\beta}\int\prod_{i=1}^{4}\mathrm{d}\tau e^{i\omega_{1}(\tau_{1}-\tau_{2})+i\nu(\tau_{2}-\tau_{3})+i\omega_{4}(\tau_{3}-\tau_{4})}\langle Tc_{a}(\tau_{1})c_{b}^{\dagger}(\tau_{2})c_{c}(\tau_{3})c_{d}^{+}(\tau_{4})\rangle
\end{eqnarray*}
So 
\begin{equation}
\chi_{abcd}^{4}(i\omega,i\omega',i\nu)=\int\mathrm{d}\tau\mathrm{d}\tau'\mathrm{d}\tau''e^{i\omega\tau+i\nu\tau''+i\omega'\tau'}\chi_{abcd}^{4}(\tau,\tau',\tau'')
\end{equation}

Note also the subtlety that the 4-point correlator has 6 spin-blocks.
Consider the single-site case 
\begin{equation}
\chi_{\sigma_{1}\sigma_{2}\sigma_{3}\sigma_{4}}^{4}(\tau_{1}-\tau_{2},\tau_{3}-\tau_{4},\tau_{2}-\tau_{3})=\langle Tc_{\sigma_{1}}(\tau_{1})c_{\sigma_{2}}^{+}(\tau_{2})c_{\sigma_{3}}(\tau_{3})c_{\sigma_{4}}^{+}(\tau_{4})\rangle
\end{equation}
So we may have $\sigma_{1}=\sigma_{2}$ and $\sigma_{3}=\sigma_{4}$
or $\sigma_{1}=\sigma_{4}$ and $\sigma_{3}=\sigma_{2}$. The latter
case one may denote with the extra bar over the spin indices. 
\begin{equation}
\chi_{\sigma\sigma'}^{4}(\tau_{1}-\tau_{2},\tau_{3}-\tau_{4},\tau_{2}-\tau_{3})=\langle Tc_{\sigma}(\tau_{1})c_{\sigma}^{+}(\tau_{2})c_{\sigma'}(\tau_{3})c_{\sigma'}^{+}(\tau_{4})\rangle
\end{equation}
\begin{equation}
\chi_{\overline{\sigma\sigma'}}^{4}(\tau_{1}-\tau_{2},\tau_{3}-\tau_{4},\tau_{2}-\tau_{3})=\langle Tc_{\sigma}(\tau_{1})c_{\sigma'}^{+}(\tau_{2})c_{\sigma'}(\tau_{3})c_{\sigma}^{+}(\tau_{4})\rangle
\end{equation}
However, the two are connected by symmetry 
\begin{equation}
\chi_{\sigma\sigma'}^{4}(\tau_{1}-\tau_{2},\tau_{3}-\tau_{4},\tau_{2}-\tau_{3})=\chi_{\overline{\sigma\sigma'}}^{4}(\tau_{1}-\tau_{4},\tau_{3}-\tau_{2},\tau_{4}-\tau_{3})
\end{equation}
For the calculation of $\chi^{3}$ this is not needed.

\section{Symmetries of correlators}

\subsection{Four-point correlator $\chi^{4}$}

We define:

\begin{equation}
\chi_{\downarrow\uparrow\,ijkl}^{4}(\tau_{1}-\tau_{2},\tau_{3}-\tau_{4},\tau_{2}-\tau_{3})\equiv\langle Tc_{\downarrow i}(\tau_{1})c_{\downarrow j}^{+}(\tau_{2})c_{\uparrow k}(\tau_{3})c_{\uparrow l}^{+}(\tau_{4})\rangle
\end{equation}
Assume now 
\begin{equation}
\beta>\tau_{1}>\tau_{2}>\tau_{3}>\tau_{4}>0\label{time_order}
\end{equation}
we get 
\begin{equation}
\langle Tc_{\downarrow i}(\tau_{1})c_{\downarrow j}^{+}(\tau_{2})c_{\uparrow k}(\tau_{3})c_{\uparrow l}^{+}(\tau_{4})\rangle=\langle c_{\downarrow i}(\tau_{1})c_{\downarrow j}^{+}(\tau_{2})c_{\uparrow k}(\tau_{3})c_{\uparrow l}^{+}(\tau_{4})\rangle\label{time_ordered}
\end{equation}
On the other hand 
\begin{equation}
\chi_{\overline{\downarrow\uparrow}\,ijkl}^{4}(\tau_{1}-\tau_{2},\tau_{3}-\tau_{4},\tau_{2}-\tau_{3})=\langle Tc_{\downarrow i}(\tau_{1})c_{\uparrow j}^{+}(\tau_{2})c_{\uparrow k}(\tau_{3})c_{\downarrow l}^{+}(\tau_{4})\rangle=\langle c_{\downarrow i}(\tau_{1})c_{\uparrow j}^{+}(\tau_{2})c_{\uparrow k}(\tau_{3})c_{\downarrow l}^{+}(\tau_{4})\rangle\label{overline_du}
\end{equation}
wher in the last step we assumed Eq.\ref{time_order}. However, for
the time-ordered average to be equal to the r.h.s. of Eq.\ref{time_ordered}
we need to map 
\begin{eqnarray}
\tau_{1}\rightarrow\tau_{1}, & i\rightarrow i\\
\tau_{2}\rightarrow\tau_{4}, & j\rightarrow l\\
\tau_{3}\rightarrow\tau_{3}, & k\rightarrow k\\
\tau_{4}\rightarrow\tau_{2}, & l\rightarrow j
\end{eqnarray}
and then 
\begin{equation}
\chi_{\overline{\downarrow\uparrow}\,ilkj}^{4}(\tau_{1}-\tau_{4},\tau_{3}-\tau_{2},\tau_{2}-\tau_{3})=\langle Tc_{\downarrow i}(\tau_{1})c_{\uparrow l}^{+}(\tau_{4})c_{\uparrow k}(\tau_{3})c_{\downarrow j}^{+}(\tau_{2})\rangle=-\langle c_{\downarrow i}(\tau_{1})c_{\downarrow j}^{+}(\tau_{2})c_{\uparrow k}(\tau_{3})c_{\uparrow l}^{+}(\tau_{4})\rangle\label{time_ordered2}
\end{equation}
Now 
\begin{eqnarray}
\tau & = & \tau_{1}-\tau_{2}\label{tau_conversion}\\
\tau' & = & \tau_{3}-\tau_{4}\nonumber \\
\tau'' & = & \tau_{2}-\tau_{3}\nonumber \\
\tau_{1} & = & \tau+\tau_{2}\nonumber \\
\tau_{4} & = & \tau_{3}-\tau'\nonumber \\
\tau_{1}-\tau_{4} & = & \tau+\tau_{2}-\tau_{3}+\tau'=\tau+\tau'+\tau''\nonumber 
\end{eqnarray}
and therefore in general 
\begin{equation}
\boxed{\chi_{\downarrow\uparrow\,ijkl}^{4}(\tau,\tau',\tau'')=-\chi_{\overline{\downarrow\uparrow}\,ilkj}^{4}(\tau+\tau'+\tau'',-\tau'',-\tau')}\label{overline_vs_no_overline}
\end{equation}

Furthermore 
\begin{equation}
\chi_{\overline{\uparrow\downarrow}\,ijkl}^{4}(\tau_{1}-\tau_{2},\tau_{3}-\tau_{4},\tau_{2}-\tau_{3})=\langle Tc_{\uparrow i}(\tau_{1})c_{\downarrow j}^{+}(\tau_{2})c_{\downarrow k}(\tau_{3})c_{\uparrow l}^{+}(\tau_{4})\rangle=\langle c_{\uparrow i}(\tau_{1})c_{\downarrow j}^{+}(\tau_{2})c_{\downarrow k}(\tau_{3})c_{\uparrow l}^{+}(\tau_{4})\rangle\label{overline_ud}
\end{equation}
where we applied time order according to Eq.\ref{time_order}. For
the time-ordered product to be the same as the r.h.s. of Eq.\ref{overline_du},
we need to map 
\begin{eqnarray}
\tau_{1}\rightarrow\tau_{3}, & i\rightarrow k\\
\tau_{2}\rightarrow\tau_{4}, & j\rightarrow l\\
\tau_{3}\rightarrow\tau_{1}, & k\rightarrow i\\
\tau_{4}\rightarrow\tau_{2}, & l\rightarrow j
\end{eqnarray}
\begin{equation}
\chi_{\overline{\uparrow\downarrow}\,klij}^{4}(\tau_{3}-\tau_{4},\tau_{1}-\tau_{2},\tau_{4}-\tau_{1})=\langle Tc_{\uparrow k}(\tau_{3})c_{\downarrow l}^{+}(\tau_{4})c_{\downarrow i}(\tau_{1})c_{\uparrow j}^{+}(\tau_{2})\rangle=\langle c_{\downarrow i}(\tau_{1})c_{\uparrow j}^{+}(\tau_{2})c_{\uparrow k}(\tau_{3})c_{\downarrow l}^{+}(\tau_{4})\rangle\label{time_ordered2}
\end{equation}
and again using Eqs.\ref{tau_conversion}, we get that in general
\begin{equation}
\boxed{\chi_{\overline{\downarrow\uparrow}\,ijkl}^{4}(\tau,\tau',\tau'')=\chi_{\overline{\uparrow\downarrow}\,klij}^{4}(\tau',\tau,-\tau-\tau'-\tau'')}\label{overlined_property}
\end{equation}
Anologously, it also holds 
\begin{equation}
\boxed{\chi_{\downarrow\uparrow\,ijkl}^{4}(\tau,\tau',\tau'')=\chi_{\uparrow\downarrow\,klij}^{4}(\tau',\tau,-\tau-\tau'-\tau'')}\label{no_overlined_property}
\end{equation}

In the case when the symmetry between the spins is preserved, we know
\begin{eqnarray}
\chi_{\sigma\sigma',ijkl}^{4} & = & \chi_{\sigma'\sigma,ijkl}^{4}\\
\chi_{\overline{\sigma\sigma'},ijkl}^{4} & = & \chi_{\overline{\sigma'\sigma},ijkl}^{4}
\end{eqnarray}
at all times $(\tau,\tau',\tau'')$. We can use this together with
properties Eq.\ref{overlined_property} and \ref{no_overlined_property}
to get a further property, that when no spin symmetry is broken

\begin{equation}
\boxed{\chi_{\overline{\sigma\sigma'}\,ijkl}^{4}(\tau,\tau',\tau'')=\chi_{\overline{\sigma\sigma'}\,klij}^{4}(\tau',\tau,-\tau-\tau'-\tau'')}\label{chi4_property}
\end{equation}
\begin{equation}
\boxed{\chi_{\sigma\sigma'\,ijkl}^{4}(\tau,\tau',\tau'')=\chi_{\sigma\sigma'\,klij}^{4}(\tau',\tau,-\tau-\tau'-\tau'')}
\end{equation}


\subsection{Three-point correlator}

First, let's look at some definitions. The pauli matrices in the $(\uparrow,\downarrow)$
basis 
\begin{equation}
\sigma_{x}=\begin{pmatrix}0 & 1\\
1 & 0
\end{pmatrix},\;\;\sigma_{y}=\begin{pmatrix}0 & -i\\
i & 0
\end{pmatrix},\;\;\sigma_{z}=\begin{pmatrix}1 & 0\\
0 & -1
\end{pmatrix}
\end{equation}
The spin operators 
\begin{eqnarray}
\mathbf{S} & = & (S^{x},S^{y},S^{z})\\
S^{I} & = & \frac{1}{2}\sum_{\sigma,\sigma'=\uparrow,\downarrow}c_{\sigma}^{\dagger}\sigma_{\sigma\sigma'}^{I}c_{\sigma'}\nonumber \\
S^{z} & = & \frac{1}{2}(n_{\uparrow}-n_{\downarrow})\nonumber \\
S^{+} & = & S^{x}+iS^{y}=c_{\uparrow}^{\dagger}c_{\downarrow}\nonumber \\
S^{-} & = & S^{x}-iS^{y}=c_{\downarrow}^{\dagger}c_{\uparrow}\nonumber \\
S^{x} & = & \frac{1}{2}(S^{+}+S^{-})\nonumber \\
S^{y} & = & \frac{1}{2i}(S^{+}-S^{-})\nonumber \\
\mathbf{S}\cdot\mathbf{S} & = & S^{x}S^{x}+S^{y}S^{y}+S^{z}S^{z}\nonumber \\
 & = & \frac{1}{4}(S^{+}S^{+}+S^{-}S^{-}+S^{+}S^{-}+S^{-}S^{+})-\frac{1}{4}(S^{+}S^{+}+S^{-}S^{-}-S^{+}S^{-}-S^{-}S^{+})+S^{z}S^{z}\nonumber \\
 & = & \frac{1}{2}(S^{+}S^{-}+S^{-}S^{+})+\frac{1}{2}(n_{\uparrow}n_{\uparrow}+n_{\downarrow}n_{\downarrow}-n_{\uparrow}n_{\downarrow}-n_{\downarrow}n_{\uparrow})\nonumber 
\end{eqnarray}

First consider 
\begin{eqnarray}
\chi_{\sigma\sigma'ijk}^{3,z}(\tau_{1}-\tau_{2},\tau_{3}-\tau_{2}) & = & \langle Tc_{\sigma i}(\tau_{1})c_{\sigma'j}^{+}(\tau_{2})S_{k}^{z}(\tau_{3})\rangle\\
 & = & \langle Tc_{\sigma i}(\tau_{1})c_{\sigma'j}^{+}(\tau_{2})(n_{\uparrow\,k}(\tau_{3})-n_{\downarrow\,k}(\tau_{3}))\rangle
\end{eqnarray}
In the absence of the spin-mixing terms one of the spin indices drops
off 
\begin{eqnarray*}
\chi_{\sigma\,ijk}^{3,z}(\tau_{1}-\tau_{2},\tau_{3}-\tau_{2}) & = & \frac{1}{2}\left(\langle Tc_{\sigma i}(\tau_{1})c_{\sigma j}^{+}(\tau_{2})n_{\uparrow\,k}(\tau_{3})\rangle-\langle Tc_{\sigma i}(\tau_{1})c_{\sigma j}^{+}(\tau_{2})n_{\downarrow\,k}(\tau_{3}))\rangle\right)\\
 & = & \frac{1}{2}\left(\chi_{\sigma,ijk}^{3,n_{\up}}(\tau_{1}-\tau_{2},\tau_{3}-\tau_{2})-\chi_{\sigma,ijk}^{3,n_{\dn}}(\tau_{1}-\tau_{2},\tau_{3}-\tau_{2})\right)
\end{eqnarray*}
where we defined 
\begin{eqnarray}
\chi_{\sigma,ijk}^{3,n_{\sigma'}}(\tau_{1}-\tau_{2},\tau_{3}-\tau_{2}) & = & \langle Tc_{\sigma i}(\tau_{1})c_{\sigma j}^{+}(\tau_{2})n_{\sigma'\,k}(\tau_{3})\rangle\\
 & = & \langle Tc_{\sigma i}(\tau_{1})c_{\sigma j}^{+}(\tau_{2})c_{\sigma'\,k}^{+}(\tau_{3})c_{\sigma'\,k}(\tau_{3})\rangle\nonumber \\
 &  & ...\;\;\mathrm{assume}\;\;\beta>\tau_{1}>\tau_{2}>\tau_{3}>0\;\;\mathrm{then\;apply\;time\;order}\nonumber \\
 & = & \langle c_{\sigma i}(\tau_{1})c_{\sigma j}^{+}(\tau_{2})(1-c_{\sigma'\,k}(\tau_{3})c_{\sigma'\,k}^{+}(\tau_{3}))\rangle\nonumber \\
 & = & G_{\sigma\,ij}(\tau_{1}-\tau_{2})+\chi_{\sigma\sigma'\,ijkk}^{4}(\tau_{1}-\tau_{2},0^{+},\tau_{2}-\tau_{3})\nonumber \\
 & = & -\chi_{\sigma\sigma'\,ijkk}^{4}(\tau_{1}-\tau_{2},0^{-},\tau_{2}-\tau_{3})\nonumber 
\end{eqnarray}
where the extra minus is due to time ordering in $\chi^{4}$. 
\begin{eqnarray}
\chi_{\up\,ijk}^{3,z}(\tau,\tau') & = & -\frac{1}{2}\left(\chi_{\upup\,ijkk}^{4}(\tau,0^{-},-\tau')-\chi_{\updn,ijkk}^{4}(\tau,0^{-},-\tau')\right)\label{chi3z_fromchi4}\\
\chi_{\dn\,ijk}^{3,z}(\tau,\tau') & = & -\frac{1}{2}\left(\chi_{\dnup\,ijkk}^{4}(\tau,0^{-},-\tau')-\chi_{\dndn,ijkk}^{4}(\tau,0^{-},-\tau')\right)
\end{eqnarray}
and when the symmetry between spins is preserved 
\begin{equation}
\boxed{\chi_{\up\,ijk}^{3,z}(\tau,\tau')=-\chi_{\dn\,ijk}^{3,z}(\tau,\tau')}\label{chi3z_fromchi4}
\end{equation}

Consider now 
\begin{equation}
\chi_{\sigma\sigma'ijk}^{3,\pm}(\tau_{1}-\tau_{2},\tau_{3}-\tau_{2})=\langle Tc_{\sigma i}(\tau_{1})c_{\sigma'j}^{+}(\tau_{2})S_{k}^{\pm}(\tau_{3})\rangle
\end{equation}
In the absence of spin-mixing terms in the action, 
\begin{equation}
\chi_{\sigma\sigma'ijk}^{3,+}(\tau_{1}-\tau_{2},\tau_{3}-\tau_{2})=\delta_{\sigma,\uparrow}\delta_{\sigma',\downarrow}\langle Tc_{\uparrow i}(\tau_{1})c_{\downarrow j}^{+}(\tau_{2})S_{k}^{+}(\tau_{3})\rangle
\end{equation}
\begin{equation}
\chi_{\sigma\sigma'ijk}^{3,-}(\tau_{1}-\tau_{2},\tau_{3}-\tau_{2})=\delta_{\sigma,\downarrow}\delta_{\sigma',\uparrow}\langle Tc_{\downarrow i}(\tau_{1})c_{\uparrow j}^{+}(\tau_{2})S_{k}^{-}(\tau_{3})\rangle
\end{equation}
and therefore we drop the spin indices and redefine 
\begin{eqnarray}
\chi_{ijk}^{3,+}(\tau_{1}-\tau_{2},\tau_{3}-\tau_{2}) & \equiv & \langle Tc_{\uparrow i}(\tau_{1})c_{\downarrow j}^{+}(\tau_{2})c_{\uparrow k}^{+}(\tau_{3}^{+})c_{\downarrow k}(\tau_{3})\rangle\\
 & = & -\chi_{\overline{\uparrow\downarrow}\,ijkk}^{4}(\tau_{1}-\tau_{2},0^{-},\tau_{2}-\tau_{3})\\
 & = & \chi_{\uparrow\downarrow\,ijkk}^{4}(\tau_{1}-\tau_{3},\tau_{3}-\tau_{2},0^{+})
\end{eqnarray}
\begin{eqnarray}
\chi_{ijk}^{3,-}(\tau_{1}-\tau_{2},\tau_{3}-\tau_{2}) & \equiv & \langle Tc_{\downarrow i}(\tau_{1})c_{\uparrow j}^{+}(\tau_{2})c_{\downarrow k}^{+}(\tau_{3}^{+})c_{\uparrow k}(\tau_{3}))\rangle\\
 & = & -\chi_{\overline{\downarrow\uparrow}\,ijkk}^{4}(\tau_{1}-\tau_{2},0^{-},\tau_{2}-\tau_{3})
\end{eqnarray}
Or 
\begin{eqnarray}
\chi_{ijk}^{3,+}(\tau,\tau') & = & -\chi_{\overline{\uparrow\downarrow}\,ijkk}^{4}(\tau,0^{-},-\tau')\label{chi3p_fromchi4}
\end{eqnarray}
\begin{eqnarray}
\chi_{ijk}^{3,-}(\tau,\tau') & = & -\chi_{\overline{\downarrow\uparrow}\,ijkk}^{4}(\tau,0^{-},-\tau')\\
 & = & -\chi_{\overline{\uparrow\downarrow}\,ijkk}^{4}(0^{-},\tau,\tau'-\tau)
\end{eqnarray}
When the psin-symmetry is preserved, one can make use of Eq.\ref{chi4_property}
to do a few more steps 
\begin{eqnarray}
\chi_{ijk}^{3,-}(\tau,\tau') & = & -\chi_{\overline{\uparrow\downarrow}\,ijkk}^{4}(0^{-},\tau,\tau'-\tau)\\
 & = & -\chi_{\overline{\uparrow\downarrow}\,ijkk}^{4}(\tau,0^{-},-\tau'+\tau-\tau)\\
 & = & -\chi_{\overline{\uparrow\downarrow}\,ijkk}^{4}(\tau,0^{-},-\tau')\\
 & = & \chi_{ijk}^{3,+}(\tau,\tau')
\end{eqnarray}
In the non-interacting case, $\chi^{3}\pm$ can be expressed with
only the disconnected term. We assume here $\beta>\tau>\tau'>0$,
and we know $G=G_{0}$ 
\begin{eqnarray}
\chi_{ijk}^{3,+}(\tau,\tau') & = & \langle c_{\up\,i}(\tau)c_{\up\,k}^{+}(\tau')c_{\dn\,k}(\tau')c_{\dn\,j}^{+}(0)\rangle\\
 & = & G_{\up,ik}(\tau-\tau')G_{\dn,kj}(\tau')\\
\chi_{ijk}^{3,-}(\bar{\tau},\bar{\tau}') & = & \langle c_{\dn\,i}(\bar{\tau})c_{\dn\,k}^{+}(\bar{\tau}')c_{\up\,k}(\bar{\tau}')c_{\up\,j}^{+}(0)\rangle\\
 & = & G_{\dn,ik}(\bar{\tau}-\bar{\tau}')G_{\up,kj}(\bar{\tau}')
\end{eqnarray}
So for the two to be the same one must have 
\begin{eqnarray}
\bar{\tau}-\bar{\tau}' & = & \tau'\\
\bar{\tau}' & = & \tau-\tau'\\
...\mathrm{which\;,means}\\
\bar{\tau}-\tau+\tau'=\tau'\Longrightarrow\bar{\tau} & = & \tau
\end{eqnarray}
but also 
\begin{eqnarray}
k & = & i\\
j & = & k
\end{eqnarray}
which is always satisfied in the single-site impurity problem, but
not always on the lattice. However, in the presence of translational
and inversion symmetry on must satisfy only 
\begin{eqnarray}
|\vec{k}-\vec{i}| & = & |\vec{j}-\vec{k}|
\end{eqnarray}
which would mean 
\begin{eqnarray}
\chi_{rr'}^{3,+}(\tau,\tau') & = & \chi_{r'r}^{3,-}(\tau,\tau-\tau')
\end{eqnarray}
{\color{red} But is this valid in the general case, with interactions?
} Consider the single-site problem. Note that in $\chi^{3,+}$ $c_{\up}^{+}$
and $c_{\dn}$ appear at the same time ($\tau_{3}$), while in $\chi^{3,-}$
they appear at different times ($\tau_{2}$ and $\tau_{1}$) respectively.
That means that a-priori, there is no reason for the two to be same
in general, unless there is some symmetry which connects $\chi_{\overline{\updn}\,ijkk}^{4}(\tau,0^{-},\tau')$
and $\chi_{\overline{\dnup}\,ijkk}^{4}(\tau,0^{-},\tau')$ in the
general case. To me it seems there is no such relation. If there is,
it needs to reduce to $\chi_{\overline{\updn}\,ijkk}^{4}(\tau,0^{-},\tau')=\chi_{\overline{\dnup}\,ijkk}^{4}(\tau,0^{-},\tau')$
in the spin-symmetric case.

We can now define 
\begin{eqnarray}
\chi_{\sigma\sigma'\,ijk}^{3,x} & = & \frac{1}{2}(\delta_{\sigma,\uparrow}\delta_{\sigma',\downarrow}\chi_{ijk}^{3,+}+\delta_{\sigma,\downarrow}\delta_{\sigma',\uparrow}\chi_{ijk}^{3,-})\\
\chi_{\sigma\sigma'\,ijk}^{3,y} & = & \frac{1}{2i}(\delta_{\sigma,\uparrow}\delta_{\sigma',\downarrow}\chi_{ijk}^{3,+}-\delta_{\sigma,\downarrow}\delta_{\sigma',\uparrow}\chi_{ijk}^{3,-})
\end{eqnarray}
{ \color{red} Important! } Even in the presence of spin symmetry,
$\chi_{\sigma\sigma'\,ijk}^{3,x}\neq\chi_{\sigma\sigma'\,ijk}^{3,y}$.

\subsection{Two-point correlators}

Let's prove in the presence of spin symmetry that 
\begin{equation}
\langle S^{z}(\tau)S^{z}(0)\rangle=\langle S^{x}(\tau)S^{x}(0)\rangle
\end{equation}
holds, and that it doesn't in general. We also know 
\begin{equation}
\langle S^{x}(\tau)S^{x}(0)\rangle=\frac{1}{4}\langle S^{+}S^{+}+S^{-}S^{-}+S^{+}S^{-}+S^{-}S^{+}\rangle
\end{equation}
In the absence of spin mixing the first two terms are zero. Also,
relabeling $\uparrow\rightarrow\downarrow$ maps $S^{+}\rightarrow S^{-}$,
so the last two terms must be the same in the presence of spin symmetry
\begin{equation}
\langle S^{x}(\tau)S^{x}(0)\rangle=\frac{1}{2}\langle S^{+}(\tau)S^{-}(0)\rangle
\end{equation}
Therefore we need to prove 
\begin{equation}
\frac{1}{2}(\langle n_{\uparrow}(\tau)n_{\uparrow}(0)\rangle-\langle n_{\uparrow}(\tau)n_{\downarrow}(0)\rangle)=\frac{1}{2}\langle S^{+}(\tau)S^{-}(0)\rangle
\end{equation}
\begin{equation}
\langle c_{\uparrow}^{+}(\tau^{+})c_{\uparrow}(\tau)c_{\uparrow}^{+}(0^{+})c_{\uparrow}(0)\rangle-\langle c_{\uparrow}^{+}(\tau^{+})c_{\uparrow}(\tau)c_{\downarrow}^{+}(0^{+})c_{\downarrow}(0)\rangle=\langle c_{\uparrow}^{+}(\tau^{+})c_{\downarrow}(\tau)c_{\downarrow}^{+}(0^{+})c_{\uparrow}(0)\rangle
\end{equation}
\begin{equation}
\chi_{\uparrow\uparrow}^{4}(0^{-},0^{-},\tau)-\chi_{\uparrow\downarrow}^{4}(0^{-},0^{-},\tau)=\chi_{\overline{\uparrow\downarrow}}^{4}(0^{-},0^{-},\tau)
\end{equation}
\begin{equation}
\chi_{\uparrow\uparrow}^{4}(0^{-},0^{-},\tau)=\chi_{\uparrow\downarrow}^{4}(0^{-},0^{-},\tau)+\chi_{\overline{\uparrow\downarrow}}^{4}(0^{-},0^{-},\tau)
\end{equation}
which is an identity, and in fact, we have in general (see Georg thesis,
Eq.2.89, {\color{red} PROVE}) 
\begin{equation}
\boxed{\chi_{\uparrow\uparrow}^{4}(\tau,\tau',\tau'')=\chi_{\uparrow\downarrow}^{4}(\tau,\tau',\tau'')+\chi_{\overline{\uparrow\downarrow}}^{4}(\tau,\tau',\tau'')}
\end{equation}
and therefore in the presence of spin symmetry, 
\begin{equation}
\boxed{-\chi_{\dn\,ijk}^{3,z}(\tau,\tau')=\chi_{\uparrow}^{3,z}(\tau,\tau')=\frac{1}{2}\chi^{3,+}(\tau,\tau')=\frac{1}{2}\chi^{3,-}(\tau,\tau')}
\end{equation}

In the ordered phase, as we know 
\[
[S^{+},S^{-}]=2S^{z}
\]
there will be a discontinuity (jump) of 
\[
\chi^{+-}(\tau-\tau')=\langle TS^{+}(\tau)S^{-}(\tau')\rangle
\]
at $\tau=0$ of size $2S_{z}$, and we will have 
\[
\chi^{+-}(\tau)\neq\chi^{+-}(-\tau)
\]
which will produce an imaginary part in the polarization $P^{+-}(i\nu)$,
and therefore in $W(i\nu)$
\end{document}
